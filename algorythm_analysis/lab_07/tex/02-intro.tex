\chapter*{Введение}
\addcontentsline{toc}{chapter}{Введение}

Словарь  -- структура данных, построенная  на  основе  пар  значений.  Первое  значение  пары -- ключ  для идентификации элементов, второе  -- собственно сам хранимый элемент. Например, в телефонном справочнике номеру  телефона  соответствует  фамилия  абонента. Существует несколько основных различных реализаций словаря: массив, двоичные деревья поиска, хеш-таблицы. Каждая из этих реализаций имеет свои минусы и плюсы, например время поиска, вставки и удаления элементов.

Со временем были разработаны различные алгоритмы поиска в словаре. В данной лабораторной работе будут рассмотрены три из них:
\begin{enumerate}
	\item поиск в АВЛ--дереве;
	\item бинарный поиск;
\end{enumerate}


\textbf{Целью} данной лабораторной работы является изучение и реализация алгоритмов поиска в словаре.\\

Для достижения поставленной цели необходимо выполнить следующие \textbf{задачи}.

\begin{enumerate}
	\item Изучение основ поиска значения в словаре.
	\item Исследование подходов к реализации двух алгоритмов поиска в словаре: поиск в АВЛ--дереве, бинарный поиск.
	\item Построение схем разработанных алгоритмов.
	\item Описание используемых структур данных.
	\item Определение средств программной реализации. 
	\item Реализация разработанных алгоритмов поиска в словаре.
	\item Проведение тестирования реализованных алгоритмов.
	\item Проведение сравнительного анализа времени выполнения двух алгоритмов поиска в словаре на основе экспериментальных данных.
	\item Проведение сравнительного анализа количества сравнений в худших и лучших случаях работы алгоритмов.
\end{enumerate}

\newpage