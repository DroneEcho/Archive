\chapter*{Заключение}
\addcontentsline{toc}{chapter}{Заключение}

В ходе выполнения лабораторной работы были решены следующие задачи:

\begin{enumerate}
	\item изучены основ поиска значения в словаре;
	\item исследованы подходы к реализации алгоритмов поиска в АВЛ--дереве, бинарный поиск;
	\item построены схем разработанных алгоритмов;
	\item определены средства программной реализации; 
	\item реализованы алгоритмы поиска в словаре;
	\item проведены тестирования реализованных алгоритмов;
	\item проведены сравнительные анализы времени выполнения двух алгоритмов поиска в словаре на основе экспериментальных данных;
	\item проведен сравнительный анализ количества сравнений в худших и лучших случаях работы алгоритмов.
\end{enumerate}

Из таблицы \ref{tabular:time1} можно сделать вывод, что бинарный поиск затрачивает меньше времени в 500 раз. 
Поиск по AVL-дереву и бинарный поиск требуют подготовки данных. 
Для первого алгоритма должно быть построено бинарное сбалансированное дерево, для бинарного поиска массив должен быть изначально отсортирован.


Лучшим случаем при нахождении элемента является его нахождение в корне AVL--дерева, или же в середине массива для бинарного поиска.
Лучшим случаем при отсутствии элемента является выявление его отсутствия при втором сравнении элементов во время выполнения алгоритма.


Худшим случаем при нахождении элемента является его нахождение на максимальной высоте AVL--дерева или на грани массива. 
Худшим случаем при отсутствии элемента является прохождение всей ветви AVL--дерева или же всех сравнений при бинарном поиске т.е. элемент должен быть за гранью разброса величин, находящихся в массиве.


При лучших случаях с нахождением элемента, логично, будет произведено только одно сравнение.
При лучших случаях с отсутствием элемента будет произведено 5 сравнений.


Из проведенного анализа можно сделать следующие выводы: количество сравнений при отсутствии элемента в массиве больше на 2. 
Количество сравнений при поиске в AVL--дереве больше, чем при бинарном поиске.


Следовательно наихудший вариант развития событий -- отсутствие элемента в массиве, при этом элемент должен быть вне диапазона чисел, присутствующих в массиве.