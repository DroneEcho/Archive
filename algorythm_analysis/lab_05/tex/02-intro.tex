\chapter*{Введение}
\addcontentsline{toc}{chapter}{Введение}

Многопоточность -- способность центрального процессора (CPU) или одного ядра в многоядерном процессоре одновременно выполнять несколько процессов или потоков, соответствующим образом поддерживаемых операционной системой. Смысл многопоточности — квазимногозадачность на уровне одного исполняемого процесса.

Параллельные вычисления часто используются для увеличения скорости выполнения программ. Однако приемы, применяемые для однопоточных машин, для параллельных могут не подходить. Конвейерная обработка данных является популярным приемом при работе с параллельными машинами.

При обработке данных также могут возникать ситуации, когда один набор данных необходимо обработать последовательно несколькими алгоритмами. В таком случае эффективно использовать конвейерную обработку данных, что позволяет на каждой следующей линии конвейера использовать данные, полученные с предыдущего этапа. \\

\textbf{Целью} данной лабораторной работы является реализация метода ковейерных вычислений.

Для достижения поставленной цели необходимо выполнить следующие \textbf{задачи}.

\begin{enumerate}
	\item Изучение основ конвейерных вычислений и конвейерной обработки данных.
	\item Исследование подходов к реализации конвейерных вычислений на основе алгоритма поиска подстроки в строке Бойера--Мура.
	\item Построение схем разработанного алгоритма.
	\item Определение средств программной реализации. 
	\item Проведение тестирования реализованного алгоритма.
	\item Реализация конвейерных вычислений с количеством линий не меньше трех.
	\item Проведение сравнительного анализа времени выполнения параллельной и последовательной реализаций конвейерных вычислений.
	\item Получение временной статистики программы, а именно время выполнения каждой ленты и время ожидания в очереди на выполнение.
\end{enumerate}
\newpage