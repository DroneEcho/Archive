\chapter{Исследовательская часть}

В данном разделе будет приведена постановка замера времени и сравнительный анализ алгоритмов на основе полученных данных.

\section{Технические характеристики}

Технические характеристики устройства, на котором выполнялся замер времени.
64-разрядная операционная система, процессор x64. Память 16 Гб. Процессор Intel(R) Core(TM) i7--4700HQ CPU @ 2.40 ГГц.

Во время замера времени ноутбук был нагружен только встроенными приложениями окружения, а также непосредственно системой тестирования.

\section{Время выполнения алгоритмов}

Алгоритмы замерялись при помощи функции $std::chrono::system\_clock$ из библиотеки $chrono$ языка $C++$. Данная функция возвращает значение в долях секунды счетчика производительности, то есть часов с наибольшим доступным разрешением для измерения короткой длительности.

Замеры времени для каждого выполнения проводились 10 раз. В качестве результата взято среднее время работы алгоритма.

\img{100mm}{1.png}{Замеры времени для параллельного конвейера}
\newpage
\img{100mm}{2.png}{Замеры времени для последовательного конвейера}
\newpage
\img{100mm}{time.png}{Замеры времени для последовательного и параллельного конвейеров}
\newpage
Как и ожидалось, параллельная реализация конвейерной обработки выигрывает по времени выполнения у линейной реализации. Как видно из таблицы, линейная реализация начинает работать в разы медленнее при 500 задачах для обработки.\\

\newpage
\section*{Вывод}

В данном разделе было проведен сравнительный анализ количества затраченного процессорного времени линейной и параллельной реализаций конвейерной обработки, а результате которого было выяснено, что параллельная реализация значительно выигрывают по скорости у линейной при количестве строк равном 500.

