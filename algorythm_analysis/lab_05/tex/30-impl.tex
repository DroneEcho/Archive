\chapter{Технологическая часть}

В данном разделе будут приведены средства реализации и листинги кода.


\section{Средства реализации}

В качестве языка программирования для реализации данной лабораторной работы был выбран язык программирования С++. 
Выбор обусловлен наличием библиотек для измерения времени, наличием инструментов для работы с потоками.


\section{Сведения о модулях программы}
Программа состоит из следующих модулей:
main.сpp --- главный файл программы,
conveyor.cpp --- файл с функциями для работы с конвейером,
moor.cpp --- файл с алгоритмом Бойера--Мура.


\section{Реализация алгоритмов}

В листингах \ref{lst:lev_mat}, \ref{lst:dlev_mat}, \ref{lst:thread}, \ref{lst:thread2}, приведены реализации алгоритмов.
\newpage
\begin{lstlisting}[label=lst:lev_mat,caption=Схема линейного конвейера.]
void parse_linear(std::string str, size_t size, bool is_print)
{
	time_now = 0;
	to_print.clear();
	std::queue<request> q1;
	std::queue<request> q2;
	std::queue<request> q3;
	
	queues_t queues = {.q1 = q1, .q2 = q2, .q3 = q3};
	
	for (size_t i = 0; i < size; i++)
	{
		request res = generate(str);
		queues.q1.push(res);
	}
	
	for (size_t i = 0; i < size; i++)
	{
		request req = queues.q1.front();
		stage1_linear(req, i + 1, is_print);
		queues.q1.pop();
		queues.q2.push(req);
		
		req = queues.q2.front();
		stage2_linear(req, i + 1, is_print);
		queues.q2.pop();
		queues.q3.push(req);
		
		req = queues.q3.front();
		stage3_linear(req, i + 1, is_print);
		queues.q3.pop();
		
		if (is_print)
		{
			for (auto& log1: req.logger)
			{
				to_print.push_back(log1);
			}
		}
	}
	if (is_print)
	{
		sort(to_print.begin(), to_print.end(), pred());
		for (auto& one: to_print)
		cout << one.data;
	}
}
\end{lstlisting}
\newpage
\begin{lstlisting}[label=lst:dlev_mat,caption=Главный поток параллельного конвейера]
void parse_parallel(std::string str, size_t size, bool is_print)
{
	to_print.clear();
	t1.resize(size + 1);
	t2.resize(size + 1);
	t3.resize(size + 1);
	
	for (size_t i = 0; i < size + 1; i++)
	{
		t1[i] = 0;
		t2[i] = 0;
		t3[i] = 0;
	}
	
	std::queue<request> q1;
	std::queue<request> q2;
	std::queue<request> q3;
	
	queues_t queues = {.q1 = q1, .q2 = q2, .q3 = q3};
	
	
	for (size_t i = 0; i < size; i++)
	{
		request res = generate(str);
		
		q1.push(res);
	}
	
	std::thread threads[THREADS];
	
	threads[0] = std::thread(stage1_parallel, std::ref(q1), std::ref(q2), std::ref(q3), is_print);
	threads[1] = std::thread(stage2_parallel, std::ref(q1), std::ref(q2), std::ref(q3), is_print);
	threads[2] = std::thread(stage3_parallel, std::ref(q1), std::ref(q2), std::ref(q3), is_print);
	
	for (int i = 0; i < THREADS; i++)
	{
		threads[i].join();
	}
	if (is_print)
	{
		sort(to_print.begin(), to_print.end(), pred());
		for (auto& one: to_print)
		cout << one.data;
	}
}
\end{lstlisting}
\newpage
\begin{lstlisting}[label=lst:thread,caption= Лента параллельного конвейера]
void stage1_parallel(std::queue<request> &q1, std::queue<request> &q2, std::queue<request> &q3, bool is_print)
{
	int task_num = 1;
	
	while(!q1.empty())
	{      
		m.lock();
		request req = q1.front();
		m.unlock();
		
		log_conveyor(req, task_num++, 1, is_print);
		
		m.lock();
		q2.push(req);
		q1.pop();
		m.unlock();
	}
}
\end{lstlisting}
\newpage
\begin{lstlisting}[label=lst:thread2,caption= Лента последовательного конвейера]
void log_linear(request &req, int task_num, int stage_num, bool is_print)
{
	std::chrono::time_point<std::chrono::system_clock> time_start, time_end;
	double start_res_time = time_now, res_time = 0;
	
	time_start = std::chrono::system_clock::now();
	if (stage_num == 1)
	{
		search(req.text, req.sub_str, req.ans);
	}
	else if (stage_num == 2)
	{
		search_detect(req.text, req.sub_str, req.inputs, req.comb);
	}
	else if (stage_num == 3)
	{
		log_in_file(req, task_num);
	}
	
	time_end = std::chrono::system_clock::now();
	
	res_time = (std::chrono::duration_cast<std::chrono::nanoseconds>
	(time_end - time_start).count()) / 1e9;
	
	time_now = start_res_time + res_time;
	
	if (is_print)
	{
		log_s tmp;
		tmp.data = "Task: "+ to_string(task_num) +", Tape: " + to_string(stage_num) + ", Start: "+ \
		to_string(start_res_time) + ", End: " + to_string(start_res_time + res_time) + "\n";
		tmp.start = start_res_time;
		tmp.end = start_res_time + res_time;
		req.logger.push_back(tmp);
	}
}
\end{lstlisting}
\section{Функциональные тесты}
В таблице \ref{tabular:functional_test} приведены функциональные тесты для алгоритма Бойера--Мура.

\begin{table}[h!]
	\captionsetup{singlelinecheck=off}\caption{\raggedright\label{tabular:functional_test} Тестирование функций}
	\begin{center}
		\begin{tabular}{c@{\hspace{7mm}}c@{\hspace{5mm}}c@{\hspace{7mm}}c@{\hspace{7mm}}}
			\hline
			Строка & Паттерн & Ожидаемый результат \\ \hline
			\vspace{4mm}
			$
				abcdefabcdef 
			$ &
			$
				abc
			$ &
			$
				0, 6
			$ \\
			\vspace{4mm}
			$
				abc 
			$ &
			$
				abc
			$ &
			$
				0
			$ \\
			\vspace{4mm}
			$
				bc de abc fab cdef 
			$ &
			$
				abc
			$ &
			$
				6
			$ \\
			\vspace{4mm}
			$
				bc de ac fab cdef 
			$ &
			$
			abc
			$ &
			
			Пустой вывод
			 \\
			\vspace{4mm}
			$
			bc de ac fab cdef 
			$ &
			
			Пустой ввод
			 &
			ERR: pattern cant be empty
			 \\
			\vspace{4mm}
			$
			hello we are test3ing\_he llahell
			$ &
			$
				hel
			$
			&
			$
				0, 29
			$\\
			\vspace{4mm}
		\end{tabular}
	\end{center}
\end{table}
Алгоритм прошел тестирование.
\section*{Вывод}

Были разработан и протестирован конвейерный алгоритм.