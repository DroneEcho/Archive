\chapter*{Заключение}
\addcontentsline{toc}{chapter}{Заключение}

В ходе выполнения лабораторной работы были решены следующие задачи:

\begin{enumerate}[label = [{*)}]
	\item изучены основы конвейерной обработки данных на примере алгоритма поиска подстроки в строке Бойера--Мура;
	\item выбран и описан метод обработки данных, которые будут сопоставлены методам конвейера;
	\item описана архитектура ПО и реализованы линейная и параллельная реализации алгоритма конвейерной обработки;
	\item произведен сравненительный анализ временных характеристик реализованных алгоритмов на экспериментальных данных.
\end{enumerate}

Из исследовательской части сделан вывод -- линейная реализация начинает работать в разы медленнее при 500 задачах для обработки.


Можно сделать вывод, что конвейерная обработка данных -- это полезный инструмент, который уменьшает время выполнения программы за счет параллельной обработки данных. 
Самым эффективным временем считается время, когда все линии конвейера работают параллельно, обрабатывая свои задачи. 
Этот метод дает выигрыш по времени в том случае, когда выполняемые задачи намного больше по времени, чем время, затрачиваемое на реализацию конвейера (работу с потоками, перекладывание из очереди в очередь и тд).
