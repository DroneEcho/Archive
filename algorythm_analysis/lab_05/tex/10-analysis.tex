\chapter{Аналитическая часть}
\section{Цель и задачи}
В данной лабораторной работе предлагается реализовать конвейерную обработку данных.


В качестве алгоритма был взят поиск подстроки в строке Бойера---Мура. 
Для каждого запроса будет сгенерирована строка со 100 символами.
Ленты будут выполнять следующие задачи:
\begin{enumerate}[]
	\item поиск подстроки в строке алгоритмом Бойера---Мура и запись числа найденных вхождений;
	\item поиск подстроки в строке алгоритмом Бойера---Мура, запись в массив индексов вхождений и расчет количества сравнений букв во время работы алгоритма;
	\item запись рассчитанных ранее данных в текстовый документ.
\end{enumerate}

\section{Конвейерная обработка данных}

Конвейер --- это устройство для непрерывного перемещения обрабатываемого изделия от одного рабочего к другому.\\

Конвейерное производство -- система поточной организации производства на основе конвейера, при которой оно разделено на простейшие короткие операции, а перемещение деталей осуществляется автоматически.\\

В терминах программирования ленты конвейера представлены функциями, выполняющими над неким набором данных операции и передающие их на следующую ленту конвейера. Моделирование конвейерной обработки сочетается с технологией многопоточного программирования --- под каждую ленту конвейера выделяется отдельный поток, все потоки работают в асинхронном режиме.\\

В данной лабораторной работе требуется выделить три задачи, которые будут последовательно обрабатываться на конвейерной ленте.
Каждая задача будет последовательно проходить три этапа обработки. Благодаря распараллеливанию можно добиться того, что бы на всех трех этапах происходила обработка элемента.


\section*{Вывод}

В данном разделе была рассмотрена предметная область работы и основополагающие материалы, которые в дальнейшем потребуются при реализации алгоритмов конвейерной обработки данных.\\
