\chapter{Технологическая часть}

В данном разделе будут приведены требования к программному обеспечению, средства реализации и листинги кода.

\section{Требования к вводу}
К вводу программы прилагаются данные требования:
\begin{enumerate}
	\item На вход подаются две строки.
	\item Буквы верхнего и нижнего регистров считаются различными.
	\item Допускается ввод пустых строк.
\end{enumerate}

\section{Требования к программе}
К программе прилагаются данные требования:
\begin{enumerate}
	\item Две пустые строки - корректный ввод, программа не должна аварийно завершаться.
	\item На выход программа должна вывести число -- расстояние Дамерау-Левенштейна, матрицу при необходимости.
	\item Программа позволяет тестировать каждый метод поиска расстояния Левенштейна, Дамерау--Левенштейна отдельно или все алгоритмы вместе.
\end{enumerate}

\section{Требования к программному обеспечению}

К программе предъявляется ряд требований:
\begin{itemize}
	\item у пользователя есть выбор алгоритма, или какой-то один, или все сразу, а также есть выбор тестирования времени;
	\item на вход подаются две строки на русском или английском языке в любом регистре;
	\item на выходе — искомое расстояние для выбранного метода (выбранных методов) и матрицы расстояний для матричных реализаций.
\end{itemize}

\section{Средства реализации}

В качестве языка программирования для реализации данной лабораторной работы был выбран язык программирования Python. 

Данный язык достаточно распространен и гибок в использовании. 

Время работы алгоритмов было замерено с помощью функции process\_time() из библиотеки time.

\section{Сведения о модулях программы}
Программа состоит из двух модулей:
\begin{enumerate}
	\item main.py - главный файл программы, в котором располагается меню;
	\item funcs.py - файл с дополнительными функциями, использующимися главными алгоритмами; 
	\item algs.py - файл со всеми алгоритмами, использующимися в программе;
	\item test.py - файл с замерами времени для графического изображения результата.
\end{enumerate}


\section{Код программы}

В листингах \ref{lst:lev_mat}, \ref{lst:dlev_rec}, \ref{lst:dlev_mat}, \ref{lst:dlev} приведены реализации алгоритмов нахождения расстояния Дамерау--Левенштейна.
\newpage
\begin{lstlisting}[label=lst:lev_mat,caption=Функция нахождения расстояния Левенштейна с использованием матрицы.]
	def levenshtein_matrix(str1, str2, output = False):
	n = len(str1)
	m = len(str2)
	
	if n == 0 or m == 0:
	if n != 0:
	return n
	if m != 0:
	return m
	return 0
	
	matrix = create_matrix(n + 1, m + 1)
	
	for i in range(1, n + 1):
	for j in range(1, m + 1):
	change = 0 
	if (str1[i - 1] != str2[j - 1]):
	change += 1
	
	matrix[i][j] = min(matrix[i - 1][j] + 1, matrix[i][j - 1] + 1, matrix[i - 1][j - 1] + change)
	
	if output:        
	print_matrix(str1, str2, matrix)
	return(matrix[-1][-1])
	
\end{lstlisting}
\newpage
\begin{lstlisting}[label=lst:dlev_rec,caption=Функция нахождения расстояния Дамерау-Левенштейна с использованием рекурсии.]
	def damerau_levenshtein_recursive(str1, str2, out_put = False):
	n = len(str1)
	m = len(str2)
	
	if n == 0 or m == 0:
	if n != 0:
	return n
	if m != 0:
	return m
	return 0
	
	change = 0
	if str1[-1] != str2[-1]:
	change += 1
	
	min_ret = min(damerau_levenshtein_recursive(str1[:n - 1], str2) + 1,
	damerau_levenshtein_recursive(str1, str2[:m - 1]) + 1,
	damerau_levenshtein_recursive(str1[:n - 1], str2[:m - 1]) + change)
	if n > 1 and m > 1 and str1[-1] == str2[-2] and str1[-2] == str2[-1]:
	min_ret = min(min_ret,
	damerau_levenshtein_recursive(str1[:n - 2], str2[:m - 2]) + 1)
	
	return min_ret
	
\end{lstlisting}
\newpage
\begin{lstlisting}[label=lst:dlev_mat,caption=Функция нахождения расстояния Дамерау-Левенштейна с использованием матрицы.]
	def damerau_levenshtein_matrix(str1, str2, output = False):
	n = len(str1)
	m = len(str2)
	
	if n == 0 or m == 0:
	if n != 0:
	return n
	if m != 0:
	return m
	return 0
	
	matrix = create_matrix(n + 1, m + 1)
	
	for i in range(1, n + 1):
	for j in range(1, m + 1):
	change = 0 
	if (str1[i - 1] != str2[j - 1]):
	change += 1
	
	matrix[i][j] = min(matrix[i - 1][j] + 1, matrix[i][j - 1] + 1, matrix[i - 1][j - 1] + change)
	
	if (i > 1 and j > 1) and str1[i-1] == str2[j-2] and str1[i-2] == str2[j-1]:
	matrix[i][j] = min(matrix[i][j], matrix[i-2][j-2] + 1)
	if output:        
	print_matrix(str1, str2, matrix)
	return(matrix[-1][-1])
	
\end{lstlisting}
\newpage
\begin{lstlisting}[label=lst:dlev,caption=Функция нахождения расстояния Дамерау-Левенштейна с использованием рекурсии и кэша.]
	def recursive(str1, str2, n, m, matrix):
	if (matrix[n][m] != -1):
	return matrix[n][m]
	
	if (n == 0):
	matrix[n][m] = m
	return matrix[n][m]
	
	if (n > 0 and m == 0):
	matrix[n][m] = n
	return matrix[n][m]
	
	delete = recursive(str1, str2, n - 1, m, matrix) + 1
	add = recursive(str1, str2, n, m - 1, matrix) + 1
	change = 0
	
	if (str1[n - 1] != str2[m - 1]):
	change = 1
	
	change = recursive(str1, str2, n - 1, m - 1, matrix) + change
	
	matrix[n][m] = min(add, delete, change)
	if n > 1 and m > 1 and str1[n - 1] == str2[m - 2] and str1[n - 2] == str2[m - 1]:
	swap = recursive(str1, str2, n - 2, m - 2, matrix) + 1
	matrix[n][m] = min(matrix[n][m], swap)
	return matrix[n][m]
	
	def damerau_levenshtein_recursive_cash(str1, str2, output = False):
	n = len(str1)
	m = len(str2)
	
	if n == 0 or m == 0:
	if n != 0:
	return n
	if m != 0:
	return m
	return 0
	
\end{lstlisting}

\section{Функциональные тесты}
В таблице \ref{tabular:functional_test} приведены функциональные тесты для алгоритмов вычисления расстояния Дамерау — Левенштейна. Все тесты пройдены успешно.


\begin{table}[htbp]
	\captionsetup{singlelinecheck=off}
	
		\caption{\raggedright\label{tabular:functional_test} Функциональные тесты}
		\resizebox{\columnwidth}{!}{
		\begin{tabular}{|c|c|c|p{3cm}|p{3cm}|p{4cm}|p{4cm}|p{4cm}|}
			\hline
			№&Строка 1&Строка 2&Ожидаемый результат&матричный Левенштейн&матричный Дамерау-Левенштейн&рекурсивный Дамерау--Левенштейн&кеш Дамерау--Левенштейн \\
			\hline
			1&скат&кот&2&2&2&2&2 \\
			\hline
			2&видео&вадео&1&1&1&1&1 \\
			\hline
			3&картина&тюна&4&4&4&4&4 \\
			\hline
			4&-&робот&5&5&5&5&5 \\
			\hline
			5&мир&-&3&3&3&3&3 \\
			\hline
			8&Срок&срок&1&1&1&1&1 \\
			\hline
			9&мера&мероприятие&8&8&8&8&8 \\
			\hline
			10&studio&stand&4&4&4&4&4 \\
			\hline
		\end{tabular}}
	
\end{table}


\section*{Вывод}

Были разработаны и протестированы алгоритмы: нахождения расстояния Левенштейна матрично, Дамерау - Левенштейна рекурсивно, с заполнением матрицы и рекурсивно с кешем.