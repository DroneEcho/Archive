\chapter{Конструкторская часть}
В этом разделе будут приведены требования к вводу и программе, а также схемы алгоритмов нахождения расстояний Левенштейна, Дамерау-Левенштейна.

\section{Алгоритмы нахождения расстояния Левенштейна, Дамерау — Левенштейна}

На рисунке 2.1 приведена схема рекурсивного алгоритма нахождения расстояния Дамерау -- Левенштейна.
На рисунке 2.2 приведена схема матричного алгоритма нахождения расстояния Левенштейна.
На рисунках 2.3, 2.4 приведена схема рекурсивного алгоритма нахождения расстояния Дамерау -- Левенштейна с кешем.
На рисунке 2.5 приведена схема матричного алгоритма нахождения расстояния Дамерау -- Левенштейна.

\svgimg{190mm}{diags-L matrix}{Схема матричного алгоритма нахождения расстояния Левенштейна}
\svgimg{170mm}{diags-DL matrix}{Схема матричного алгоритма нахождения расстояния Дамерау--Левенштейна}
\svgimg{170mm}{diags-DL recursive}{Схема рекурсивного алгоритма нахождения расстояния Дамерау--Левенштейна с кешем}
\svgimg{190mm}{diags-Recursive}{Схема функции recursive алгоритма нахождения расстояния Дамерау--Левенштейна с кешем}
\svgimg{160mm}{diags-DL cash}{Схема рекурсивного алгоритма нахождения расстояния с кешем Дамерау--Левенштейна}

\newpage
\section*{Вывод}
Перечислены требования к вводу и программе, а также на основе теоретических данных, полученных из аналитического раздела были построены схемы требуемых алгоритмов.


