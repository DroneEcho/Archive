\chapter{Аналитическая часть}
\section{Цель и задачи}
Целью данной лабораторной работы является: Изучение метода динамического программирования на материале расстояний Левенштейна и Дамерау-Левенштейна.
Для достижения цели следует поставить следующте задачи: 
\begin{enumerate}
	\item изучение алгоритмов Левенштейна, Дамерау-Левенштейна нахождения расстояния между строками;
	\item разработка и реализация алгоритмов поиска расстояний Левенштейна в форме: матричной; Дамерау-Левенштейна в формах: матричной, рекурсивной, рекурсивной с кешем;
	\item сравнительный анализ алгоритмов определения расстояния между строками по затрачиваемому времени, памяти;
	\item выполнить замеры процессорного времени работы реализаций алгоритмов; 
	\item провести анализ полученных результатов в отчете.
\end{enumerate}

Далее этом разделе будут представлены описания алгоритмов нахождения расстояний Левенштейна, Дамерау-Левенштейна и их практическое применение.
\section{Алгоритм нахождения расстояния Левенштейна}

\textbf{Расстояние Левенштейна} - это мера разницы двух строк символов, определяемая как минимальное количество операций вставки, удаления и замены, необходимых для перевода одной строки в другую.

Расстояние Левенштейна может быть найдено по формуле \ref{eq:l}, которая задана как
\begin{equation}
	\label{eq:l}
	\begin{aligned}
		d_{a,b}(i, j) = \begin{cases}
			\max(i, j), ~~\text{если }\min(i, j) = 0,\\
			\min \lbrace \\
			\qquad d_{a,b}(i, j-1) + 1,\\
			\qquad d_{a,b}(i-1, j) + 1,\\
			\qquad d_{a,b}(i-1, j-1) + m(a[i], b[j])\\
			\rbrace
		\end{cases},
	\end{aligned}
\end{equation}

\section{Алгоритм нахождения расстояния 
	Дамерау-Левенштейна}

\textbf{Расстояние Дамерау-Левенштейна} - это мера разницы двух строк символов, определяемая как минимальное количество операций вставки, удаления, замены и транспозиции (перестановки двух соседних символов), необходимых для перевода одной строки в другую. Является модификацией расстояния Левенштейна: к операциям вставки, удаления и замены символов, определённых в расстоянии Левенштейна добавлена операция транспозиции (перестановки) символов.

Расстояние Дамерау — Левенштейна может быть найдено по формуле \ref{eq:d}, которая задана как
\begin{equation}
	\label{eq:d}
	d_{a,b}(i, j) = \begin{cases}
		\max(i, j), ~~\text{если }\min(i, j) = 0,\\
		\min \lbrace \\
		\qquad d_{a,b}(i, j-1) + 1,\\
		\qquad d_{a,b}(i-1, j) + 1,\\
		\qquad d_{a,b}(i-1, j-1) + m(a[i], b[j]), ~~\text{иначе}\\
		\qquad \left[ \begin{array}{cc}d_{a,b}(i-2, j-2) + 1, &\text{если }i,j > 1;\\
			\qquad &\text{}a[i] = b[j-1]; \\
			\qquad &\text{}b[j] = a[i-1]\\
			\qquad \infty, & \text{иначе}\end{array}\right.\\
		\rbrace
	\end{cases},
\end{equation}

\section{Рекурсивный алгоритм нахождения 
	расстояния Дамерау-Левенштейна}

Рекурсивный алгоритм нахождения расстояния Дамерау-Левенштейна выполняет прогон для данных, которые еще не были обработаны. Строится дерево из сравняемых букв, и ,доходя до максимальной глубины, запрос возвращается рекурсивно, выбирая минимальное из возможных преобразований. Результат нахождения расстояния возвращается рекурсивно к исходному запросу.

\section{Рекурсивный алгоритм нахождения расстояния Дамерау-Левенштейна с использованием кеша}

Рекурсивный алгоритм заполнения можно оптимизировать по времени выполнения с использованием кеша. В качестве кеша используется матрица. Суть данной оптимизации заключается в последовательном заполнении матрицы при выполнении рекурсии. 
В случае, если рекурсивный алгоритм выполняет прогон для данных, которые еще не были обработаны, результат нахождения расстояния заносится в матрицу. В случае, если обработанные ранее данные встречаются снова, для них расстояние не находится и алгоритм переходит к следующему шагу.

\section*{Вывод}

Формулы Левенштейна, Дамерау — Левенштейна для рассчета расстояния между строками задаются рекурентно, а следовательно, алгоритмы могут быть реализованы рекурсивно или итерационно.