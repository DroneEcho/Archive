\chapter*{Введение}
\addcontentsline{toc}{chapter}{Введение}

\textbf{Расстояние Левенштейна}  (редакционное расстояние, дистанция редактирования) — метрика, измеряющая разность между двумя последовательностями символов. Она определяется как минимальное количество односимвольных операций (вставки, удаления, замены), необходимых для превращения одной строки в другую. В общем случае, операциям, используемым в этом преобразовании, можно назначить разные цены. Широко используется в теории информации и компьютерной лингвистике.

\textbf{Расстояние Левенштейна} и его обобщения активно применяются для:
\begin{itemize}
	\item исправления ошибок в слове (в поисковых системах, базах данных, при вводе текста, при автоматическом распознавании отсканированного текста или речи);
	\item сравнения текстовых файлов утилитой \code{diff} и ей подобными (здесь роль «символов» играют строки, а роль «строк» — файлы);
	\item в биоинформатике для сравнения генов;
	
\end{itemize}

\textbf{Расстояние Дамерау — Левенштейна} (названо в честь учёных Фредерика Дамерау и Владимира Левенштейна) — это мера разницы двух строк символов, определяемая как минимальное количество операций вставки, удаления, замены и транспозиции (перестановки двух соседних символов), необходимых для перевода одной строки в другую. Является модификацией расстояния Левенштейна, так как к операциям вставки, удаления и замены символов, определённых в расстоянии Левенштейна добавлена операция транспозиции (перестановки) символов.

\newpage