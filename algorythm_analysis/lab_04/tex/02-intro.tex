\chapter*{Введение}
\addcontentsline{toc}{chapter}{Введение}

Рост числа сложных задач, решение которых связано с применением современных информационных технологий, ведёт к необходимости использования параллельных вычислений. Параллельные вычисления носят междисциплинарный характер. Они затрагивают, в частности, такие области, как численные методы, структуры и алгоритмы обработки данных, аппаратное и программное обеспечение, системный анализ. Это позволяет применять знания, полученные при исследовании параллельных вычислений, в различных сферах научно-практической деятельности.


Параллельные вычисления --- современная многогранная область вычислительных наук, бурно развивающаяся и являющаяся наиболее актуальной в ближайшие десятилетия. Актуальность данной области складывается из множества факторов, и в первую очередь, исходя из потребности в больших вычислительных ресурсах для решения прикладных задач моделирования процессов в физике, биофизике, химии и др. К тому же, традиционные последовательные архитектуры вычислителей и схем вычислений находятся на пороге технологического предела. В то же время технологический прорыв в области создания средств межпроцессорных и межкомпьютерных коммуникаций позволяет реализовать одно из ключевых звеньев параллелизма --- эффективное управление в распределении вычислений по различным компонентам интегрированной вычислительной установки.


Целью данной работы является  исследование и последующая реализация многопоточности на примере алгоритма Бойера--Мура поиска подстроки в строке (поиск одной и той же подстроки в файле от 100 Мбайт).


Для достижения поставленной цели необходимо решить следующие задачи:
\begin{enumerate}
	\item исследовать алгоритм Бойера--Мура поиска подстроки в строке;
	\item составить схему алгоритма;
	\item определить средства программной реализации алгоритма;
	\item реализовать алгоритм Бойера-Мура с использованием параллельных процессов и без;
	\item протестировать разработанное программное обеспечение;
	\item исследовать влияние количества потоков на время работы программы.
\end{enumerate}

\newpage