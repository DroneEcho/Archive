\chapter{Технологическая часть}

В данном разделе будут приведены средства реализации и листинги кода.


\section{Средства реализации}

В качестве языка программирования для реализации данной лабораторной работы был выбран язык программирования Python. Выбор обусловлен наличием библиотек для измерения времени, наличием инструментов для работы с параллельными потоками. Были использованы библиотеки threading, multiprocessing, os.


\section{Сведения о модулях программы}
Программа состоит из следующих модулей:
main.py - главный файл программы, в котором располагается вся программа,
timer.py - файл программы c замерами времени.


\section{Реализация алгоритмов}

В листингах \ref{lst:lev_mat}, \ref{lst:dlev_rec}, \ref{lst:dlev_mat}, \ref{lst:thread}, \ref{lst:thread2}, приведены реализации алгоритмов.
\newpage
\begin{lstlisting}[label=lst:lev_mat,caption=Алгоритм Бойера--Мура поиска подстроки в строке.]
	def bouer_moor(text, pattern):
		bad_char = bad_char_heuristic(pattern)
		good_suffix = good_suffix_heuristic(pattern)
		i = 0
		indexes = []
		while i <= len(text) - len(pattern):
			j = len(pattern) - 1
			while j >= 0 and pattern[j] == text[i + j]:
				j -= 1
			if j < 0:
				indexes.append(i)
				i += good_suffix[0]
			else:
				x = j - bad_char.get(text[i + j], -1)
				y = good_suffix[j + 1]
				i += max(x, y)
		return indexes
		
\end{lstlisting}
\newpage
\begin{lstlisting}[label=lst:dlev_rec,caption= Эвристика хорошего суффикса]
	def good_suffix_heuristic(pattern):
		good_suffix = [-1] * (len(pattern) + 1)
		border = [0] * (len(pattern) + 1)
		i = len(pattern)
		j = len(pattern) + 1
		border[i] = j
		while i > 0:
			while j <= len(pattern) and pattern[i - 1] != pattern[j - 1]:
				if good_suffix[j] == 0:
					good_suffix[j] = j - i
			j = border[j]
			i -= 1
			j -= 1
		border[i] = j
		j = border[0]
		for i in range(len(pattern) + 1):
			if good_suffix[i] == -1:
				good_suffix[i] = j
			if i == j:
				j = border[j]
	return good_suffix
\end{lstlisting}
\newpage
\begin{lstlisting}[label=lst:dlev_mat,caption=Эвристика плохого символа]
	def bad_char_heuristic(pattern):
		bad_char = {}
		for i in range(len(pattern)):
			bad_char[pattern[i]] = i
		return bad_char
\end{lstlisting}
\newpage
\begin{lstlisting}[label=lst:thread,caption= Схема рабочего потока алгоритм Бойера--Мура]
def bouer_moor_thread(thread_results_pipe, text, pattern, good_suffix, bad_char, start, end):
	indexes = []
	pattern_len = len(pattern)
	i = start
	text_len = len(text) if end + pattern_len > len(text) else end + pattern_len
	
	while i <= text_len - pattern_len:
		j = pattern_len - 1
		while j >= 0 and pattern[j] == text[i + j]:
			j -= 1
		if j < 0:
			indexes.append(i)
			i += good_suffix[0]
		else:
			x = j - bad_char.get(text[i + j], -1)
			y = good_suffix[j + 1]
		if (max(x, y) > text_len - pattern_len and max(x, y) < text_len):
			i = text_len - pattern_len
		else:
			i += max(x, y)
	thread_results_pipe.send(indexes)
\end{lstlisting}
\newpage
\begin{lstlisting}[label=lst:thread2,caption= Cхема главного потока (диспетчера) алгоритма Бойера--Мура]
def bouer_moor_parallel(text, pattern, thread_num):
	thread_results_pipe,thread_results_pipe_recieve = Pipe()
	thread_results = []
	bad_char = bad_char_heuristic(pattern)
	good_suffix = good_suffix_heuristic(pattern)
	text_len = len(text)
	
	text_slice = int(text_len / thread_num)
	
	k = 0
	
	threads = []
	for i in range(thread_num):
		k_end = k + text_slice
		if (i == thread_num - 1):
			k_end = text_len
		threads.append(threading.Thread(target=bouer_moor_thread, args=(thread_results_pipe, text, pattern, \
						good_suffix, bad_char, k, k_end,)))
		k += text_slice
	for thread in threads:
		thread.start()
	
	for thread in threads:
		thread_results.append(thread_results_pipe_recieve.recv())
		thread.join()
	
	return thread_results
\end{lstlisting}
\section{Функциональные тесты}
В таблице \ref{tabular:functional_test} приведены функциональные тесты для алгоритма Бойера--Мура.

\begin{table}[h!]
	\captionsetup{singlelinecheck=off}\caption{\raggedright\label{tabular:functional_test} Тестирование функций}
	\begin{center}
		\begin{tabular}{c@{\hspace{7mm}}c@{\hspace{5mm}}c@{\hspace{7mm}}c@{\hspace{7mm}}}
			\hline
			Строка & Паттерн & Ожидаемый результат \\ \hline
			\vspace{4mm}
			$
				abcdefabcdef 
			$ &
			$
				abc
			$ &
			$
				0, 6
			$ \\
			\vspace{4mm}
			$
				abc 
			$ &
			$
				abc
			$ &
			$
				0
			$ \\
			\vspace{4mm}
			$
				bc de abc fab cdef 
			$ &
			$
				abc
			$ &
			$
				6
			$ \\
			\vspace{4mm}
			$
				bc de ac fab cdef 
			$ &
			$
			abc
			$ &
			
			Пустой вывод
			 \\
			\vspace{4mm}
			$
			bc de ac fab cdef 
			$ &
			
			Пустой ввод
			 &
			ERR: pattern cant be empty
			 \\
			\vspace{4mm}
			$
			hello we are test3ing\_he llahell
			$ &
			$
				hel
			$
			&
			$
				0, 29
			$\\
			\vspace{4mm}
		\end{tabular}
	\end{center}
\end{table}
Алгоритм прошел проверку с последовательным и параллельным запуском.
\section*{Вывод}

Были разработан и протестирован алгоритм Бойера--Мура.