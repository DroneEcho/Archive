\chapter{Аналитическая часть}
\section{Алгоритм Бойера--Мура поиска подстроки в строке}

Алгоритм Бойера--Мура считается наиболее быстрым среди алгоритмов общего назначения, предназначенных для поиска подстроки в строке. Важной особенностью алгоритма является то, что он выполняет сравнения в шаблоне справа налево в отличие от многих других алгоритмов.


Алгоритм сравнивает символы шаблона $x$ справа налево, начиная с самого правого, один за другим с символами исходной строки $y$.
Если символы совпадают, производится сравнение предпоследнего символа шаблона и так до конца.
Если все символы шаблона совпали с наложенными символами строки, значит, подстрока найдена, и поиск окончен.
В случае несовпадения какого-либо символа (или полного совпадения всего шаблона) он использует две предварительно вычисляемые функции, чтобы сдвинуть позицию для начала сравнения вправо.


Таким образом для сдвига позиции начала сравнения алгоритм Бойера-Мура выбирает между двумя эвристическими функциями, называемыми эвристиками хорошего суффикса и плохого символа. Так как функции эвристические, то выбор между ними простой — ищется такое итоговое значение, чтобы не проверялось максимальное число позиций и при этом найдены все подстроки, равные шаблону. В итоге, эвристика стоп-символа -- последнее вхождение стоп-символа в подстроку, эвристика безопасного (совпавшего) суффикса -- для каждого возможного суффикса $S$ шаблона указываем наименьшую величину, на которую нужно сдвинуть вправо шаблон, чтобы он снова совпал.


Если же какой-то символ образца не совпадает с соответствующим символом строки, образец сдвигается на величину сдвига вправо, и проверка снова начинается с последнего символа.


В качестве величины сдвига берется большее из двух значений:
1) Значение, полученное с помощью таблицы стоп-символов по простому правилу:
Если несовпадение произошло на позиции $i$, а стоп-символ «c», то значение величины сдвига будет равно $i - StopTable[c]$, где 
$StopTable[c]$ -- значение, записанное в таблице стоп-символов, для символа «c».
2) Значение, полученное из таблицы суффиксов.

\section{Параллельный алгоритм Бойера--Мура}
Так как при работе алгоритма отступ после каждого сравнения может быть рассчитан только после нахождения несоответствия в тексте, был выбран подход разделения
текста, в котором происходит поиск паттерна, на равные части. Вычисление результата для каждой части будет независим от вычисления результата других частей. При ситуации, где разделение текста происходит на паттерне,  выбран подход, где параллельный процесс с собственной частью может получить доступ к следующей за ней части текста ровно на размер строки паттерна.

\section*{Вывод}

В данном разделе был рассмотрен алгоритм Бойера--Мура и возможность его оптимизации с помощью распараллеливания потоков.