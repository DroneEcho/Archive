\chapter{Конструкторская часть}
В данном разделе приводится схема рассматриваемого алгоритма, определяются требования к программе.

\section{Требования к программе}
К вводу программы прилагаются данные требования:
На вход подается название файла, строка, вхождения которой ищутся в тексте файла. 


На выходе программа выводит индексы вхождений строки в тексте. 


Требования к программе:
\begin{enumerate}
	\item Корректное нахождение всех вхождений паттерна внутри текста; 
	\item Программа не должна аварийно завершаться при вводе неверного файла;
	\item При вводе пустого паттерна программа не должна начинать поиск.
\end{enumerate}
\section{Разработка алгоритмов}

На рисунке \ref{svgimg:diag-moor} приведена схема алгоритма Бойера--Мура поиска подстроки в строке.
На рисунке \ref{svgimg:diag-good_suffix} приведена схема подпрограммы поиска хорошего суффикса.
На рисунке \ref{svgimg:diag-bad_char} приведена схема подпрограммы поиска плохого символа.
На рисунке \ref{svgimg:diag-moor_thread} приведена схема алгоритма Бойера--Мура поиска подстроки в строке для параллельных процессов.
На рисунке \ref{svgimg:diag-threading} приведена cхема главного потока (диспетчера) алгоритма Бойера--Мура поиска подстроки в строке.


\svgimg{200mm}{diag-moor}{Схема алгоритма Бойера--Мура поиска подстроки в строке}
\svgimg{200mm}{diag-good_suffix}{Схема подпрограммы поиска хорошего суффикса}
\svgimg{200mm}{diag-bad_char}{Схема подпрограммы поиска плохого символа}
\svgimg{200mm}{diag-moor_thread}{Схема рабочего потока алгоритма Бойера--Мура поиска подстроки в строке}
\svgimg{200mm}{diag-threading}{Схема главного потока (диспетчера) алгоритма Бойера--Мура поиска подстроки в строке}

\newpage

\section*{Вывод}
В данном разделе была рассмотрена схема алгоритма  Бойера--Мура, были приведены требования к программе.

