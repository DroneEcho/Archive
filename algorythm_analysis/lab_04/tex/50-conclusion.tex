\chapter*{Заключение}
\addcontentsline{toc}{chapter}{Заключение}

В ходе выполнения лабораторной работы были решены следующие задачи:

\begin{enumerate}
	\item исследован алгоритм Бойера--Мура поиска подстроки в строке;
	\item составлены схемы алгоритма Бойера--Мура, параллельного алгоритма Бойера--Мура;
	\item определены средства программной реализации алгоритма;
	\item реализован алгоритм Бойера-Мура с использованием параллельных процессов и без;
	\item протестировано разработанное программное обеспечение;
	\item исследовано влияние количества потоков на время работы программы.
\end{enumerate}


Использование многопоточности дает существенный прирост эффективности. Однако, при использовании большего количества потоков(более 16), прирост эффективности падает до несущественных размеров. Последующее увеличение потоков нецелесообразно. 


Стоит учесть, что одновременное выполнение большого количества потоков может тратить много процессорного времени на переключение контекста за счет фактической обработки, сбрасывая блоки памяти в файловую систему и увеличивая количество операций ввода/вывода, что в конечном итоге замедляет работу всего приложения и засоряет хост-машину\ref{lib:proc}.


Также возникает проблема в работе с памятью. Все потоки используют одну и ту же память процесса, но если каждый поток требует больше памяти, потоки будут ограничены памятью процесса\ref{lib:proc}. 


Для устройства, на котором проводился эксперимент, оптимальное количество потоков для эффективной работы был выбран 16-поточный алгоритм.