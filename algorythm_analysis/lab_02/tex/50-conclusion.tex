\chapter*{Заключение}
\addcontentsline{toc}{chapter}{Заключение}

В ходе выполнения лабораторной работы были решены следующие задачи:

\begin{enumerate}
	\item Были изучены и реализованы 3 алгоритма перемножения матриц: обычный, Копперсмита -- Винограда, оптимизированный Копперсмита -- Винограда.
	\item Был произведён анализ трудоёмкости алгоритмов;
	\item Выполнены замеры процессорного времени работы реализаций алгоритмов;
	\item Был сделан сравнительный анализ алгоритмов на основе экспериментальных данных.
\end{enumerate}


Можно сделать вывод, что выбор алгоритма умножения матриц зависит от размера матриц и требуемой временной эффективности. Алгоритм Винограда занимает больше памяти, чем стандартный алгоритм перемножения из-за использования дополнительных векторов при вычислении. Алгоритм Копперсмита - Винограда имеет меньшую трудоемкость и выполняется в среднем в 1.3 раза быстрее, чем обычный алгоритм умножения матриц. Учлучшенный алгоритм Копперсмита - Винограда работает в 1.5 раз быстрее cтандартного умножения матриц и в 1.2 быстрее оригинального алгоритма Винограда. 