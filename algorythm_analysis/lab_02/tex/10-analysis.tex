\chapter{Аналитическая часть}
\section{Цель и задачи}
Целью данной лабораторной работы является: Изучение алгоритмов умножения матриц Копперсмита-Винограда, оптимизированного Копперсмита-Винограда.
Для достижения цели следует поставить следующие задачи: 

\begin{enumerate}
	\item Реализация трёх алгоритмов умножения матриц: обычный, Копперсмита-Винограда, оптимизированный Копперсмита-Винограда.
	\item Выполнить замеры процессорного времени работы реализаций алгоритмов;
	\item Сравнительный анализ трудоёмкости алгоритмов на основе теоретических расчетов и выбранной модели вычислений.
	\item Сравнительный анализ алгоритмов на основе экспериментальных данных.
\end{enumerate}

\newpage
\section{Стандартный алгоритм умножения матриц}

Пусть даны две прямоугольные матрицы
\begin{equation}
	A_{lm} = \begin{pmatrix}
		a_{11} & a_{12} & \ldots & a_{1m}\\
		a_{21} & a_{22} & \ldots & a_{2m}\\
		\vdots & \vdots & \ddots & \vdots\\
		a_{l1} & a_{l2} & \ldots & a_{lm}
	\end{pmatrix},
	\quad
	B_{mn} = \begin{pmatrix}
		b_{11} & b_{12} & \ldots & b_{1n}\\
		b_{21} & b_{22} & \ldots & b_{2n}\\
		\vdots & \vdots & \ddots & \vdots\\
		b_{m1} & b_{m2} & \ldots & b_{mn}
	\end{pmatrix},
\end{equation}

тогда матрица $C$
\begin{equation}
	C_{ln} = \begin{pmatrix}
		c_{11} & c_{12} & \ldots & c_{1n}\\
		c_{21} & c_{22} & \ldots & c_{2n}\\
		\vdots & \vdots & \ddots & \vdots\\
		c_{l1} & c_{l2} & \ldots & c_{ln}
	\end{pmatrix},
\end{equation}

где
\begin{equation}
	\label{eq:M}
	c_{ij} =
	\sum_{r=1}^{m} a_{ir}b_{rj} \quad (i=\overline{1,l}; j=\overline{1,n})
\end{equation}

будет называться произведением матриц $A$ и $B$.
Стандартный алгоритм реализует данную формулу.

\section{Алгоритм Копперсмита -- Винограда}

Если посмотреть на результат умножения двух матриц, то видно, что каждый элемент в нем представляет собой скалярное произведение соответствующих строки и столбца исходных матриц.
Можно заметить также, что такое умножение допускает предварительную обработку, позволяющую часть работы выполнить заранее.

Рассмотрим два вектора $V = (v_1, v_2, v_3, v_4)$ и $W = (w_1, w_2, w_3, w_4)$.
Их скалярное произведение равно: $V \cdot W = v_1w_1 + v_2w_2 + v_3w_3 + v_4w_4$, что эквивалентно (\ref{for:new}):
\begin{equation}
	\label{for:new}
	V \cdot W = (v_1 + w_2)(v_2 + w_1) + (v_3 + w_4)(v_4 + w_3) - v_1v_2 - v_3v_4 - w_1w_2 - w_3w_4.
\end{equation}

Несмотря на то, что второе выражение требует вычисления б\'{о}льшего количества операций, чем стандартный алгоритм умножения матриц, оно предлагает возможность предварительной обработки. Это означает, что некоторые значения могут быть вычислены заранее и сохранены для последующего использования. В результате каждый элемент матрицы будет получен с помощью только двух умножений и пяти сложений, а также сложения с предварительно посчитанными суммами соседних элементов текущих строк и столбцов.


Таким образом, алгоритм Копперсмита — Винограда может быть более эффективным на практике, поскольку операция сложения выполняется быстрее операции умножения в современных компьютерах. Это позволяет сократить общее количество операций и уменьшить время выполнения алгоритма.

\section{Оптимизированный алгоритм Копперсмита -- Винограда}

Оптимизированный алгоритм Винограда представляет собой обычный алгоритм Винограда, за исключением следующих оптимизаций:

\begin{itemize}

\item Используется битовый сдвиг вместо умножения на 2
\item Заменена операция x = x + k на x += k 
\item Предвычисляются слагаемые для алгоритма.
\end{itemize}

\section*{Вывод}

В данном разделе были рассмотрены алгоритмы классического умножения матриц и алгоритм Винограда: обычный и оптимизированный.