\chapter*{Введение}
\addcontentsline{toc}{chapter}{Введение}

Алгоритм Копперсмита — Винограда является алгоритмом умножения матриц и был разработан с целью снижения сложности этой операции. Он основан на использовании техники предварительных вычислений, которая позволяет уменьшить количество операций умножения.


Алгоритм Копперсмита — Винограда, впервые предложенный в 1987 году Д. Копперсмитом и Ш. Виноградом, имеет асимптотическую сложность $O(n^{2,3755})$, где $n$ - размер стороны матрицы.


Основная идея алгоритма заключается в том, чтобы предварительно вычислить некоторые значения, которые будут использоваться при последующем умножении матриц. Алгоритм состоит из следующих шагов:
\begin{enumerate}
	\item Инициализация: создание матриц для хранения предварительных вычислений и определение размеров матриц, которые будут умножаться.
	\item Предварительные вычисления: для каждой строки первой матрицы и каждого столбца второй матрицы вычисляются некоторые промежуточные значения, которые будут использоваться при финальном умножении. Эти значения сохраняются в отдельных матрицах.
	\item Умножение: используя предварительно вычисленные значения, производится финальное умножение матриц.
	\item Возврат результата: полученная после умножения матрица возвращается как результат операции.
\end{enumerate}


Алгоритм Копперсмита — Винограда имеет лучшую асимптотическую сложность по сравнению с другими известными алгоритмами умножения матриц, однако он не используется в практике из-за высокой константы пропорциональности. Он становится эффективным только для матриц большого размера, которые превышают память современных компьютеров.

\newpage