\chapter{Технологическая часть}

В данном разделе будут приведены требования к программному обеспечению, средства реализации и листинги кода.

\section{Требования к вводу}
К вводу программы прилагаются данные требования:
\begin{enumerate}
	\item Перед вводом матрицы запрашиваются ее размерности.
	\item На вход подаются две матрицы.
	\item Ввод матрицы - числа типа int.
\end{enumerate}

\section{Требования к программе}
К программе прилагаются данные требования:

В вводе размерностей (n1, n2, m1, m2) n2 обязана равняться m1. На выход программа должна вывести три итоговые матрицы, рассчитанные разными методами.


\section{Требования к программному обеспечению}

К программе предъявляется ряд требований:
\begin{itemize}
	\item у пользователя есть выбор алгоритма, или какой-то один, или все сразу;
	\item на вход подаются две матрицы с содержанием -- числа типа int;
	\item на выходе — матрица с результатом умножения исходных матриц.
\end{itemize}

\section{Средства реализации}

В качестве языка программирования для реализации данной лабораторной работы был выбран язык программирования С. Выбор обусловлен наличием библиотек для измерения времени, наличием инструментов для работы с матрицами.


\section{Сведения о модулях программы}
Программа состоит из следующих модулей:
main.c - главный файл программы, в котором располагается вся программа,
time.c - файл программы c замерами времени.


\section{Реализация алгоритмов}

В листингах \ref{lst:lev_mat}, \ref{lst:dlev_rec}, \ref{lst:dlev_mat},  приведены реализации алгоритмов умножения матриц.
\newpage
\begin{lstlisting}[label=lst:lev_mat,caption=Стандартный алгоритм умножения матриц.]
	int multMatrix(int **matrix1, int **matrix2, int **matrix_res, size_t n1, size_t m1, size_t m2){
		for (size_t i = 0; i < n1; i++) {
			for (size_t j = 0; j < m2; j++) {
				matrix_res[i][j] = 0;
				for (size_t k = 0; k < m1; k++) {
					matrix_res[i][j] += matrix1[i][k] * matrix2[k][j];
				}
			}
		}
		return 0;
	}
	
\end{lstlisting}
\newpage
\begin{lstlisting}[label=lst:dlev_rec,caption=Алгоритм Винограда]
int Vinograd(int **matrix1, int **matrix2, int **matrix_res, size_t n1, size_t m1, size_t m2)
{
    int *rowCnt = (int*)malloc(n1 * sizeof(int));
	int *colCnt = (int*)malloc(m2 * sizeof(int));;
	for (size_t i = 0; i < n1; i++)
	{
		rowCnt[i] = 0;
		for (size_t j = 0; j < m1 / 2; j++)
		rowCnt[i] = rowCnt[i] + matrix1[i][j * 2] * matrix1[i][(j*2) + 1];
	}
	for (size_t i = 0; i < m2; i++) {
		colCnt[i] = 0;
		for (size_t j = 0; j < m1 / 2; j++)
		colCnt[i] = colCnt[i] + matrix2[j * 2][i] * matrix2[j*2 + 1][i];
	}
	
	for (size_t i = 0; i < n1; i++) {
		for (size_t j = 0; j < m2; j++) {
			matrix_res[i][j] = -rowCnt[i] - colCnt[j];
			for (size_t k = 0; k < m1 / 2; k++) {
				matrix_res[i][j] += (matrix1[i][2*k] + matrix2[2*k+1][j]) * \
				(matrix1[i][2*k+1] + matrix2[2*k][j]);
			}
		}
	}
	if (m1 % 2 == 1){
		for (size_t i = 0; i < n1; i++)
			for (size_t j = 0; j < m2; j++)
				matrix_res[i][j] = matrix_res[i][j] + matrix1[i][m1 - 1] * matrix2[m1 - 1][j];
	}
	free(rowCnt);
	free(colCnt);
	return 0;
}
	
\end{lstlisting}
\newpage
\begin{lstlisting}[label=lst:dlev_mat,caption= Оптимизированный Алгоритм Винограда]
	int Vinograd_Opt(int **matrix1, int **matrix2, int **matrix_res, size_t n1, size_t m1, size_t m2)
	{
		int *rowCnt = (int*)malloc(n1 * sizeof(int));
		int *colCnt = (int*)malloc(m2 * sizeof(int));;
		for (size_t i = 0; i < n1; i++) {
			rowCnt[i] = 0;
			for (int j = 0; j < m1 / 2; j++)
				rowCnt[i] += matrix1[i][j << 1] * matrix1[i][(j<<1) + 1];
		}
		for (size_t i = 0; i < m2; i++) {
			colCnt[i] = 0;
			for (int j = 0; j < m1 / 2; j++)
				colCnt[i] += matrix2[j * 2][i] * matrix2[(j<<1) + 1][i];
		}
		for (size_t i = 0; i < n1; i++) {
			for (size_t j = 0; j < m2; j++) {
				matrix_res[i][j] = -rowCnt[i] - colCnt[j];
				for (int k = 0; k < m1 / 2; k++) {
					matrix_res[i][j] += (matrix1[i][k<<1] + matrix2[(k<<1)+1][j]) * \
					(matrix1[i][2*k+1] + matrix2[2*k][j]);
				}
			}
		}
		if (m1 % 2 == 1) {
			for (size_t i = 0; i < n1; i++)
			for (size_t j = 0; j < m2; j++)
			matrix_res[i][j] += matrix1[i][m1 - 1] * matrix2[m1 - 1][j];
		}
		free(rowCnt);
		free(colCnt);
		return 0;
	}
	
\end{lstlisting}
\newpage

\section{Функциональные тесты}
В таблице \ref{tabular:functional_test} приведены функциональные тесты для алгоритмов умножения матриц.

\begin{table}[h!]
	\captionsetup{singlelinecheck=off}\caption{\raggedright\label{tabular:functional_test} Тестирование функций}
	\begin{center}
		\begin{tabular}{c@{\hspace{7mm}}c@{\hspace{7mm}}c@{\hspace{7mm}}c@{\hspace{7mm}}c@{\hspace{7mm}}c@{\hspace{7mm}}}
			\hline
			Первая матрица & Вторая матрица & Ожидаемый результат \\ \hline
			\vspace{4mm}
			$\begin{pmatrix}
				1 & 2 & 3\\
				4 & 5 & 6\\
				7 & 8 & 9
			\end{pmatrix}$ &
			$\begin{pmatrix}
				1 & 2 & 3\\
				4 & 5 & 6\\
				7 & 8 & 9
			\end{pmatrix}$ &
			$\begin{pmatrix}
				30 & 36 & 42 \\
				66 & 81 & 96 \\
				102 & 126 & 150
			\end{pmatrix}$ \\
			\vspace{2mm}
			\vspace{2mm}
			$\begin{pmatrix}
				1 & 2 & 3\\
				4 & 5 & 6
			\end{pmatrix}$ &
			$\begin{pmatrix}
				1 & 2\\
				3 & 4\\
				5 & 6
			\end{pmatrix}$ &
			$\begin{pmatrix}
				22 & 28\\
				49 & 64
			\end{pmatrix}$ \\
			\vspace{2mm}
			\vspace{2mm}
			$\begin{pmatrix}
				2
			\end{pmatrix}$ &
			$\begin{pmatrix}
				2
			\end{pmatrix}$ &
			$\begin{pmatrix}
				4
			\end{pmatrix}$ \\
			\vspace{2mm}
			\vspace{2mm}
			$\begin{pmatrix}
				1 & -2 & 3\\
				4 & 5 & -6\\
				-7 & 8 & 9
			\end{pmatrix}$ &
			$\begin{pmatrix}
				1 & -2 & 3\\
				4 & 5 & -6\\
				-7 & 8 & 9
			\end{pmatrix}$ &
			$\begin{pmatrix}
				-28 & 12 & 42\\
				66 & -31 & -72\\
				-38 & 126 & 12
			\end{pmatrix}$\\
			\vspace{2mm}
			\vspace{2mm}
			$\begin{pmatrix}
				1 & 2\\
				3 & 4\\
				5 & 6
			\end{pmatrix}$ &
			$\begin{pmatrix}
				1 & 2 & 3\\
				4 & 5 & 6
			\end{pmatrix}$ &
			$\begin{pmatrix}
				22 & 28\\
				49 & 64
			\end{pmatrix}$ \\
			\vspace{2mm}
			\vspace{2mm}
			$\begin{pmatrix}
				0 & 1\\
				0 & 1
			\end{pmatrix}$ &
			$\begin{pmatrix}
				0 & 1 \\
				0 & 1 
			\end{pmatrix}$ &
			$\begin{pmatrix}
				0 & 1\\
				0 & 1
			\end{pmatrix}$ \\
			\vspace{2mm}
			\vspace{2mm}
			$\begin{pmatrix}
				4 \\
				4 \\
				4 
			\end{pmatrix}$ &
			$\begin{pmatrix}
				-4 & -4 & -4
			\end{pmatrix}$ &
			$\begin{pmatrix}
				-48
			\end{pmatrix}$ \\
			\vspace{2mm}
			\vspace{2mm}
			$\begin{pmatrix}
				20 & 27\\
				19 & 16
			\end{pmatrix}$ &
			$\begin{pmatrix}
				-2 & -3\\
				-5 & -7
			\end{pmatrix}$ &
			$\begin{pmatrix}
				-175 & -249\\
				-118 & -169
			\end{pmatrix}$ \\
			\vspace{2mm}
			\vspace{2mm}
			$\begin{pmatrix}
				0
			\end{pmatrix}$ &
			$\begin{pmatrix}
				0
			\end{pmatrix}$ &
			$\begin{pmatrix}
				0
			\end{pmatrix}$ \\
			\vspace{2mm}
			\vspace{2mm}
		\end{tabular}
	\end{center}
\end{table}
Все алгоритмы прошли проверку.
\section*{Вывод}

Были разработаны и протестированы алгоритмы: стандартный алгоритм умножения матриц, Алгоритм Винограда, оптимизированный Алгоритм Винограда.