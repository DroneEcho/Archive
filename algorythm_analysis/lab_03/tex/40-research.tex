\chapter{Исследовательская часть}

В данном разделе будет приведена постановка эксперимента и сравнительный анализ алгоритмов на основе полученных данных.

\section{Технические характеристики}

Технические характеристики устройства, на котором выполнялся эксперимент по замеру времени:

\begin{itemize}
	\item[---] 64-разрядная операционная система, процессор x64;
	\item[---] память 16 Гб;
	\item[---] процессор Intel(R) Core(TM) i7--4700HQ CPU @ 2.40 ГГц.
\end{itemize}

Во время выполнения эксперимента ноутбук был нагружен только встроенными приложениями окружения, а также непосредственно программой.
\section{Расчет затрат памяти}
Пусть с -- размер сортируемого массива, эксперименттогда затраты памяти
на приведенные выше алгоритмы будут следующими:


Сортировка перемешиванием:


-- Массив -- с * sizeof(int);


-- Переменные -- 5 * sizeof(int).


Сортировка Шелла:


--  Массив -- с * sizeof(int);


-- Переменные -- 5 * sizeof(int).


Плавная сортировка:


-- Массив -- с * sizeof(int);


-- Переменные -- 3 * sizeof(int).


В итоге, сортировка перемешиванием и сортировка Шелла занимают одинаковое количество памяти. Плавная сортировка выигрывает по памяти всего на размер двух переменных типа int.
\section{Время выполнения алгоритмов}

Время выполнения алгоритмов было замерено при помощи функции clock() из библиотеки <time.h> языка C. Данная функция возвращает количество временных тактов с момента запуска программы, вызвавшей фун­кцию clock().

Замеры времени для каждой реализации сортировки проводились 100 раз. В качестве результата взято среднее время работы реализации алгоритма.

Результаты замеров приведены в таблицах \ref{tab:time}, \ref{tab:time1}, \ref{tab:time2} (время в мс).

\begin{table}[!ht]
	\captionsetup{singlelinecheck=off}
	\caption{\raggedright\label{tab:time}Результаты замеров времени на случайных данных(мс.)}
	\begin{tabular}{|c|p{3cm}|p{3cm}|p{4.2cm}|
		}
		
		\hline
		Размерность массива & Сортировка перемешиванием& Сортировка Шелла & Плавная сортировка\\
		\hline
		500 & 0.000 480 & 0.000 020 & 0.000 010 \\
		\hline
		1000 & 0.001 510 & 0.000 110 &0.000 090 \\
		\hline
		1500 & 0.003 550 & 0.000 070 & 0.000 120 \\
		\hline
		2000 & 0.005 940 & 0.000 120 & 0.000 170 \\
		\hline
		2500 & 0.009 570 & 0.000 220 & 0.000 200 \\
		\hline
		3000 & 0.014 710 & 0.000 230 & 0.000 140 \\
		\hline
		3500 & 0.019 260 & 0.000 300 & 0.000 210 \\
		\hline
		4000 & 0.026 680 & 0.000 330 & 0.000 370 \\
		\hline
		4500 & 0.033 320 & 0.000 470 & 0.000 300 \\
		\hline
		5000 & 0.040 930 & 0.000 470 & 0.000 390 \\
		\hline
	\end{tabular}
\end{table}

На изображении \ref{img:rand} приведен график на основе табличных данных из \ref{tab:time}.
\img{90mm}{rand}{График результата замеров времени на случайных данных}
\newpage
Как видно из данных с случайным заполнением массива, сортировка перемешиванием заняла больше времени чем остальные сортировки с таким объемом неотсортированных данных; плавная сортировка и сортировка Шелла значительно эффективнее по затраченному времени работы сортировки перемешиванием, плавная сортировка быстрее остальных.

\begin{table}[!ht]
	\captionsetup{singlelinecheck=off}
	\caption{\raggedright\label{tab:time1}Результаты замеров времени на отсортированных данных(мс.)}
	\begin{tabular}{|c|p{3cm}|p{3cm}|p{4.2cm}|
		}
		
		\hline
		Размерность массива & Сортировка перемешиванием& Сортировка Шелла & Плавная сортировка\\
		\hline
		500 & 0.000 020 & 0.000 000 & 0.000 010 \\
		\hline
		1000 & 0.000 030 & 0.000 050 &0.000 010 \\
		\hline
		1500 & 0.000 040 & 0.000 070 & 0.000 030 \\
		\hline
		2000 & 0.000 080 & 0.000 070 & 0.000 060 \\
		\hline
		2500 & 0.000 110 & 0.000 140 & 0.000 070 \\
		\hline
		3000 & 0.000 160 & 0.000 160 & 0.000 070 \\
		\hline
		3500 & 0.000 210 & 0.000 190 & 0.000 070 \\
		\hline
		4000 & 0.000 290 & 0.000 180 & 0.000 120 \\
		\hline
		4500 & 0.000 360 & 0.000 220 & 0.000 120 \\
		\hline
		5000 & 0.000 460 & 0.000 240 & 0.000 150 \\
		\hline
	\end{tabular}
\end{table}
\newpage
На изображении \ref{img:sorted} приведен график на основе табличных данных из \ref{tab:time1}.
\img{120mm}{sorted}{График результата замеров времени на отсортированных данных}
\newpage
Как видно из данных с отсортированным заполнением массива, плавная сортировка быстрее всех сортировок справлялась с данными.


\begin{table}[!ht]
	\captionsetup{singlelinecheck=off}
	\caption{\raggedright\label{tab:time2}Результаты замеров времени на отсортированных в обратном порядке данных(мс.)}
	\begin{tabular}{|c|p{3cm}|p{3cm}|p{4.2cm}|
		}
		
		\hline
		Размерность массива & Сортировка перемешиванием& Сортировка Шелла & Плавная сортировка\\
		\hline
		500 & 0.000 390 & 0.000 020 & 0.000 010 \\
		\hline
		1000 & 0.001 540 & 0.000 090 &0.000 030 \\
		\hline
		1500 & 0.003 280 & 0.000 120 & 0.000 040 \\
		\hline
		2000 & 0.005 990 & 0.000 120 & 0.000 050 \\
		\hline
		2500 & 0.009 460 & 0.000 140 & 0.000 050 \\
		\hline
		3000 & 0.013 000 & 0.000 160 & 0.000 060 \\
		\hline
		3500 & 0.017 650 & 0.000 200 & 0.000 090 \\
		\hline
		4000 & 0.023 050 & 0.000 200 & 0.000 150 \\
		\hline
		4500 & 0.028 870 & 0.000 240 & 0.000 220 \\
		\hline
		5000 & 0.036 060 & 0.000 250 & 0.000 220 \\
		\hline
	\end{tabular}
\end{table}
\newpage
На изображении \ref{img:rev} приведен график на основе табличных данных из \ref{tab:time2}.
\img{90mm}{rev}{График результата замеров времени на отсортированных в обратном порядке данных}
Как видно из данных с отсортированным в обратном порядке заполнением массива, сортировка перемешиванием справляется медленнее всех из рассматриваемых алгоритмов с данным объемом данных. Сортировка Шелла и Плавная сортировка в 150 раз быстрее сортировки перемешиванием на замерах от 500 вхождений в сортируемом массиве до 5000, плавная сортировка выигрывает по времени у сортировки Шелла.
\newpage
\section*{Вывод}
Из проведенного анализа сделаны следующие выводы: все три сортировки демонстрируют высокую производительность при работе с уже отсортированными данными, в каждой сортировке осуществляется только один проход по циклам; сортировка перемешиванием заняла больше всего времени при отсортированных в обратном порядке данных $O(n^2)$, в отличии от плавной сортировки $O(N\log{N})$, которая была самой быстрой среди рассматриваемых; во время анализа замеров времени при случайных данных сортировка перемешиванием также заняла больше всего времени, плавная сортировка -- лидер по эффективности выполнения.

