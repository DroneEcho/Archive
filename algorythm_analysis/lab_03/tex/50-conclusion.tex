\chapter*{Заключение}
\addcontentsline{toc}{chapter}{Заключение}

В ходе выполнения лабораторной работы были решены следующие задачи:

\begin{itemize}
	\item[---] были реализованы алгоритмы сортировки Шелла, премешиванием, плавная;
	\item[---] был произведён анализ трудоёмкости алгоритмов;
	\item[---] был сделан сравнительный анализ алгоритмов на основе экспериментальных данных по времени, памяти.
\end{itemize}


Выбор алгоритма сортировки зависит от факторов, являющихся критичными в поставленной задаче. Анализ этих факторов помогает выбрать наиболее подходящий алгоритм сортировки для конкретной задачи.


Из исследовательской части сделаны следующие выводы: все три сортировки демонстрируют высокую производительность при работе с уже отсортированными данными, в каждой сортировке осуществляется только один проход по циклам. Сортировка перемешиванием заняла больше всего времени при отсортированных в обратном порядке данных $O(n^2)$, в отличии от плавной сортировки $O(N\log{N})$, которая была самой быстрой среди рассматриваемых. Во время анализа замеров времени при случайных данных сортировка перемешиванием также заняла больше всего времени, плавная сортировка -- лидер по быстроте выполнения. Сортировка перемешиванием и сортировка Шелла занимают одинаковое количество памяти. Плавная сортировка выигрывает по памяти всего на размер двух переменных типа int.