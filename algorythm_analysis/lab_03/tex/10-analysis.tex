\chapter{Аналитическая часть}
\section{Цель и задачи}
Целью данной работы является сравнение трех алгоритмов сортировки: плавная (smoothsort), сортировка перемешиванием и сортировка Шелла.


Для достижения поставленной цели необходимо решить следующие задачи:
\begin{itemize}
	\item[---] реализовать перечисленные алгоритмы сортировки;
	\item[---] рассчитать трудоемкость реализованных алгоритмов;
	\item[---] провести сравнительный анализ реализаций алгоритмов по затраченному процессорному времени, памяти.
\end{itemize}

\newpage
\section{Плавная сортировка}

Плавная сортировка является модификацией алгоритма сортировки кучей. Данная сортировка сортирует массив путем создания кучи и многократного извлечения максимального элемента.


Основная идея плавной сортировки заключается в том, что кучи определяются  с помощью чисел Леонардо, которые задаются следующим образом:
\begin{equation}
	\begin{aligned}
		L(n) = \begin{cases}
			\quad1, ~~\text{если }\quad n = 0,\\
			\quad1, ~~\text{если }\quad n = 1,\\
			\quad L(n - 1) + L(n - 2) + 1, ~~\text{если }\min n > 1.\\
		\end{cases}
	\end{aligned}
\end{equation}


\textbf{K-ая куча Леонардо} -- это двоичное дерево с количеством вершин L(k), удовлетворяющее следующим условиям:
\begin{itemize}
\item[--] число, записанное в корне, не меньше чисел в поддеревьях;
\item[--] левым поддеревом является (k - 1)-я куча Леонардо;
\item[--] правым -- (k - 2)-я куча Леонардо.
\end{itemize}

Алгоритм плавной сортировки работает следующим образом.
\begin{enumerate}
\item В массиве записаны элементы, которые надо отсортировать.
\item Превращение массива в последовательность куч, кучи получены с помощью чисел Леонардо.
\item Повторяются шаги 2 и 3 до тех пор, пока расстояние не станет равным 0.
\item Пока последовательность куч не пустая, достаем максимальный элемент (это всегда корень самой правой кучи) и восстанавливаем порядок куч, который мог измениться.
\end{enumerate}




\section{Cортировка перемешиванием}
Шейкерная сортировка, также известная как сортировка перемешиванием или сортировка коктейлем, является модификацией сортировки пузырьком.


Основная идея шейкерной сортировки заключается в том, чтобы проходить по массиву элементов в обоих направлениях, меняя местами соседние элементы, если они находятся в неправильном порядке. Это позволяет перемещать большие элементы в конец массива и малые элементы в начало массива, ускоряя процесс сортировки.


Алгоритм шейкерной сортировки работает следующим образом.
\begin{enumerate}
\item Устанавливается начальное значение флага, указывающего, были ли выполнены обмены на текущей итерации. Изначально флаг устанавливается в значение "ложь".
\item Выполняются следующие действия:
	\begin{itemize}
		\item[--] рассматриваются все элементы массива, начиная с первого и до последнего элемента;
		\item[--] cравниваются пары элементов, и если они находятся в неправильном порядке, они меняются местами, а флаг устанавливается в значение $True$;
		\item[--] рассматриваются все элементы массива в обратном порядке, начиная с последнего и до первого элемента;
		\item[--] сравниваются пары элементов, и если они находятся в неправильном порядке, они меняются местами, а флаг устанавливается в значение $True$.
	\end{itemize}
\item Если флаг остается $False$ после выполнения шага 2, значит массив уже отсортирован и сортировка завершается. Иначе повтор шага 2.
\end{enumerate}

\section{Cортировка Шелла}

Сортировка Шелла является алгоритмом сортировки, который представляет собой модификацию сортировки вставками.


Основная идея сортировки Шелла заключается в том, чтобы предварительно сортировать элементы массива, находящиеся на определенном расстоянии друг от друга, а затем постепенно уменьшать это расстояние и повторять процесс сортировки. Это позволяет перемещать элементы на более короткие расстояния и ускоряет процесс сортировки.


Каждый проход в алгоритме характеризуется смещением $h_{i}$, таким, что сортируются элементы отстающие друг от друга на $h_{i}$ позиций. Шелл предлагал использовать $h_{t}=N/2$, $h_{t-1}=h_{t}/2$, …, $h_0=1$. Возможны и другие смещения, но $h_0=1$
всегда.


Алгоритм плавной сортировки работает следующим образом:
\begin{enumerate}
	\item Шаг 0. $i$ = $t$.
	\item Шаг 1. Разобьем массив на списки элементов, отстающих друг от друга на $h_{i}$. Таких списков будет $h_{i}$.
	\item Шаг 2. Отсортируем элементы каждого списка сортировкой вставками.
	\item Шаг 3. Объединим списки обратно в массив. Уменьшим $i$. Если $i$ неотрицательно -- вернемся к шагу 1.
\end{enumerate}

\section*{Вывод}

В данном разделе были рассмотрены алгоритмы сортировки: плавная (smoothsort), сортировка перемешиванием и сортировка Шелла.