\chapter{Технологическая часть}

В данном разделе будут приведены средства реализации программы и листинги кода.

\section{Средства реализации}

В качестве языка программирования для реализации данной лабораторной работы был выбран язык программирования С. Выбор обусловлен наличием библиотек для измерения времени, наличием инструментов для работы с массивами.


\section{Сведения о модулях программы}
Программа состоит из следующих модулей:
main.c - главный файл программы, в котором располагается вся программа,
time.c - файл программы c замерами времени.


\section{Реализация алгоритмов}

В листингах \ref{lst:lev_mat}, \ref{lst:dlev_rec}, \ref{lst:dlev_mat},  приведены реализации алгоритмов сортировки массивов.
\newpage
\begin{lstlisting}[label=lst:lev_mat,caption=Плавная сортировка.]
	void SmoothSort(int *arr, int c) {
		int gap = 1;
		while (gap < c) {
			gap = gap * 3 + 1;
		}
		
		while (gap > 1) {
			gap = (gap - 1) / 3;
			
			for (int i = gap; i < c; i++) {
				int temp = arr[i];
				int j;
				
				for (j = i; j >= gap && arr[j - gap] > temp; j -= gap) {
					arr[j] = arr[j - gap];
				}
				
				arr[j] = temp;
			}
		}
	}
	
\end{lstlisting}
\newpage
\begin{lstlisting}[label=lst:dlev_rec,caption=Сортировка перемешиванием]
	void ShakerSort(int *to_sort, int c)
	{
		int left = 0, right = c - 1;
		int flag = 1, t;
		while ((left < right) && flag > 0) {
			flag = 0;
			for (int i = left; i < right; i++) {
				if (to_sort[i] > to_sort[i + 1])
				{
					t = to_sort[i];
					to_sort[i] = to_sort[i + 1];
					to_sort[i + 1] = t;
					flag = 1;
				}
			}
			right--;
			for (int i = right; i > left; i--) {
				if (to_sort[i - 1] > to_sort[i])
				{
					int t = to_sort[i];
					to_sort[i] = to_sort[i - 1];
					to_sort[i - 1] = t;
					flag = 1;
				}
			}
			left++;
		}
	}
	
\end{lstlisting}
\newpage
\begin{lstlisting}[label=lst:dlev_mat,caption=Сортировка Шелла]
	void ShellSort(int *to_sort, int c)
	{
		int i, j, step;
		int tmp;
		for (step = c / 2; step > 0; step /= 2)
		for (i = step; i < c; i++) {
			tmp = to_sort[i];
			for (j = i; j >= step; j -= step)
			{
				if (tmp < to_sort[j - step])
				to_sort[j] = to_sort[j - step];
				else
				break;
			}
			to_sort[j] = tmp;
		}
	}
\end{lstlisting}
\newpage

\section{Функциональные тесты}
В таблице \ref{tabular:functional_test} приведены функциональные тесты для алгоритмов сортировки массивов.

\begin{table}[h!]
	\captionsetup{singlelinecheck=off}\caption{\raggedright\label{tabular:functional_test} Тестирование функций}
	\begin{center}
		\begin{tabular}{c@{\hspace{7mm}}c@{\hspace{7mm}}c@{\hspace{7mm}}c@{\hspace{7mm}}}
			\hline
			Массив & Ожидаемый результат \\ \hline
			\vspace{4mm}
			$\begin{pmatrix}
				1 & 2 & 3 & 4 & 5 & 6 & 7 & 8 & 9 
			\end{pmatrix}$ &
			$\begin{pmatrix}
				1 & 2 & 3 & 4 & 5 & 6 & 7 & 8 & 9
			\end{pmatrix}$ \\
			\vspace{4mm}
			$\begin{pmatrix}
				9 & 8 & 7 & 6 & 5 & 4 & 3 & 2 & 1 
			\end{pmatrix}$ &
			$\begin{pmatrix}
				1 & 2 & 3 & 4 & 5 & 6 & 7 & 8 & 9
			\end{pmatrix}$ \\
			\vspace{4mm}
			$\begin{pmatrix}
				1 & 1 & 1 & 1 & 1 & 1 & 1 & 1 & 1 
			\end{pmatrix}$ &
			$\begin{pmatrix}
				1 & 1 & 1 & 1 & 1 & 1 & 1 & 1 & 1
			\end{pmatrix}$ \\
			\vspace{4mm}
			$\begin{pmatrix}
				1 & 1 & 2 & 2 & 8 & 8 & 3 & 3 & -1 & 4
			\end{pmatrix}$ &
			$\begin{pmatrix}
				-1 & 1 & 1 & 2 & 2 & 3 & 3 & 4 & 8 & 8
			\end{pmatrix}$ \\
			\vspace{4mm}
			$\begin{pmatrix}
				1 & 1 & 2 & 2 & 8 & 8 & 3 & 3 & -1 & -2
			\end{pmatrix}$ &
			$\begin{pmatrix}
				-2 & -1 & 1 & 1 & 2 & 2 & 3 & 3 & 8 & 8
			\end{pmatrix}$ \\
			\vspace{4mm}
			$\begin{pmatrix}
				-1 & -3 & -5 & -2
			\end{pmatrix}$ &
			$\begin{pmatrix}
				-5 & -3 & -2 & -1
			\end{pmatrix}$ \\
			\vspace{4mm}
			$\begin{pmatrix}
				-1 
			\end{pmatrix}$ &
			$\begin{pmatrix}
				-1
			\end{pmatrix}$ \\  
			
		\end{tabular}
	\end{center}
\end{table}
Тестирование пройдено успешно всеми реализациями алгоритмов.
\section*{Вывод}

Были разработаны и протестированы алгоритмы: плавная сортировка, сортировка перемешиванием, сортировка Шелла.