\chapter*{Заключение}
\addcontentsline{toc}{chapter}{Заключение}

В ходе выполнения лабораторной работы были решены следующие задачи:

\begin{enumerate}
	\item исследованы задачи коммивояжера;
	\item исследованы подходы к решению задачи коммивояжера;
	\item построены схемы разрабатываемых алгоритмов; 
	\item реализованы алгоритмы для решения задачи коммивояжера;
	\item проведены тестирования муравьиного алгоритма;
	\item проведен сравнительный анализ времени выполнения реализованных алгоритмов решения задачи коммивояжера. 
\end{enumerate}

Из проведенного анализа можно сделать следующие выводы: использование муравьиного алгоритма более выгодно с точки зрения временных затрат, тогда как алгоритм полного перебора эффективен только при малом количестве вершин графа (2 -- 8). 


На основе проведенной параметризации были выявлены оптимальные параметры для муравьиного алгоритма: a = 0.75, p = 0.75, tmax (количество дней) = 350. Однако стоит учитывать, что чем больше значение tmax, тем больше вероятность того, что будет найден идеальный маршрут, но при этом будет возрастать время выполнения программы. Также был проведен сравнительный анализ времени выполнения двух алгоритмов для решения задачи коммивояжера: муравьиный алгоритм значительно выигрывает по времени выполнения у алгоритма полного перебора при количестве вершин в рассматриваемом графе больше 9-ти. При небольших размерах матриц и в случае, если необходимо получить наименьшее
расстояние, рационально использовать алгоритм полного перебора.