\chapter{Исследовательская часть}

В данном разделе будет приведена постановка замера времени и сравнительный анализ алгоритмов на основе полученных данных.

\section{Технические характеристики}

Технические характеристики устройства, на котором выполнялся замер времени:

Операционная система: 64-разрядная операционная система, процессор x64.


Память: 16 Гб.


Процессор: Intel(R) Core(TM) i7--4700HQ CPU @ 2.40 ГГц.


Во время замера времени ноутбук был нагружен только встроенными приложениями окружения, а также непосредственно системой тестирования.

\section{Время выполнения алгоритмов}

Алгоритмы замерялись при помощи функции $time.perf\_counter()$ из библиотеки $time$ языка $Python$. Данная функция возвращает значение в долях секунды счетчика производительности, то есть часов с наибольшим доступным разрешением для измерения короткой длительности.

Замеры времени для каждого выполнения проводились 10 раз. В качестве результата взято среднее время работы алгоритма.

По результатам эксперимента видно, что муравьиный алгоритм начинает значительно выигрывать по времени у алгоритма полного перебора при количестве вершин графа начиная с 9.

Таким образом можно сделать вывод, что использовать муравьиный алгоритм для решения задачи коммивояжера выгодно с точки зрения времени выполнения, в сравнении с алгоритмом полного перебора, в случае если в анализируемом графе вершин больше либо равно 9.

\img{100mm}{time}{Замеры времени для муравьиного алгоритма в сравнении с алгоритмом, использующим полный перебор\label{img:6}}

\newpage

\section{Параметризация муравьиного алгоритма}

В муравьином алгоритме вычисления производятся на основе настраиваемых параметров. Проведем параметризацию муравьиного алгоритма для матрицы $10\times10$:\\

$\begin{pmatrix}
	0 & 3 & 6 & 5 & 3 & 1 & 1 & 7 & 8 & 3\\
	2 & 0 & 5 & 8 & 1 & 1 & 3 & 2 & 9 & 7\\
	5 & 3 & 0 & 4 & 9 & 8 & 3 & 3 & 2 & 6\\
	2 & 3 & 4 & 0 & 9 & 2 & 3 & 7 & 8 & 8\\
	6 & 2 & 5 & 9 & 0 & 6 & 7 & 2 & 2 & 3\\
	6 & 1 & 4 & 2 & 6 & 0 & 6 & 3 & 10 & 9\\
	7 & 5 & 3 & 3 & 7 & 6 & 0 & 6 & 3 & 2\\
	1 & 6 & 2 & 7 & 2 & 3 & 6 & 0 & 4 & 1\\
	1 & 7 & 9 & 8 & 2 & 10 & 3 & 4 & 0 & 9\\
	3 & 8 & 2 & 8 & 3 & 9 & 2 & 1 & 9 & 0\\
\end{pmatrix}$

\vspace{\baselineskip}

В каждом эксперименте фиксировались значения alpha, beta, p и количество дней. При этом, соблюдалось условие, что alpha+beta=const. В течение экспериментов значения alpha менялись от 0 до 1 с шагом 0.25, значения р менялись от 0 до 1 с шагом 0.25, количество дней - от 50 до 400 с шагом 50. Количество повторов каждого эксперимента равнялось 100, результатом проведения эксперимента считалась усредненная разница между длиной маршрута, рассчитанного алгоритмом полного перебора и муравьиным алгоритмом с текущими параметрами.\\

\img{100mm}{exp}{Результат эксперимента с изменением параметров}

В результате усреднения массового эксперимента было полученр, что наилучшим значением настроенного параметра alpha является значение, равное 0.75, а коэффициента испарения p — 0.25. При количестве дней равном 310 достигается наименьшее различие с минимальным путем.



\section*{Вывод}
Из проведенного анализа можно сделать следующие выводы: использование муравьиного алгоритма более выгодно с точки зрения временных затрат, тогда как алгоритм полного перебора эффективен только при малом количестве вершин графа (2 -- 8). 


На основе проведенной в данном разделе параметризации были выявлены оптимальные параметры для муравьиного алгоритма: a = 0.75, p = 0.75, tmax (количество дней) = 350. Однако стоит учитывать, что чем больше значение tmax, тем больше вероятность того, что будет найден идеальный маршрут, но при этом будет возрастать время выполнения программы. Также был проведен сравнительный анализ времени выполнения двух алгоритмов для решения задачи коммивояжера: муравьиный алгоритм значительно выигрывает по времени выполнения у алгоритма полного перебора при количестве вершин в рассматриваемом графе больше 9-ти. При небольших размерах матриц и в случае, если необходимо получить наименьшее
расстояние, рационально использовать алгоритм полного перебора.

