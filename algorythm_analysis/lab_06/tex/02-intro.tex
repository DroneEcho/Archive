\chapter*{Введение}
\addcontentsline{toc}{chapter}{Введение}

Задача коммивояжера -- одна из самых известных задач
комбинаторной оптимизации, заключающаяся в поиске самого выгодного маршрута, проходящего через указанный города хотя бы по одному разу с последующим возвратом в исходный город, относится к числу транс вычислительных: уже при относительно небольшом числе городов (66 и более) она не может быть решена методом перебора вариантов никаким теоретически мыслимыми компьютерами за время, меньшее нескольких миллиардов лет.

Муравьиный алгоритм -- алгоритм для нахождения приближенных решений задачи коммивояжера, а также решения аналогичных задач поиска маршрутов на графах. Суть подхода заключается в анализе и использовании модели поведения муравьев, ищущих пути от колонии к источнику питания, и представляет собой метаэвристическую оптимизацию.\\

\textbf{Целью} данной лабораторной работы является исследование и реализация муравьиного алгоритма на примере задачи коммивояжера.\\

Для достижения поставленной цели необходимо выполнить следующие \textbf{задачи}:

\begin{enumerate}
	\item изучение задачи коммивояжера;
	\item исследование подходов к решению задачи коммивояжера;
	\item построение схем разрабатываемых алгоритмов; 
	\item реализация алгоритмов для решения задачи коммивояжера;
	\item проведение тестирования муравьиного алгоритма;
	\item проведение сравнительного анализа времени выполнения реализованных алгоритмов решения задачи коммивояжера. 
\end{enumerate}

\newpage