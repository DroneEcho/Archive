\chapter{Аналитическая часть}
В данном разделе будет рассмотрено теоретическое описание задачи коммивояжера и методы ее решения.

\section{Задача коммивояжера}

Задача коммивояжера -- важная задача транспортной логистики, отрасли, занимающейся планированием транспортных перевозок. Коммивояжёру, чтобы распродать нужные и не очень нужные в хозяйстве товары, следует объехать n пунктов и в конце концов вернуться в исходный пункт. Требуется определить наиболее выгодный маршрут объезда. В качестве меры выгодности маршрута может служить суммарное время в пути, суммарная стоимость дороги, или, в простейшем случае, длина маршрута.\\


Ниже, на рис.\ref{d:task}, визуализирована задача коммивояжера.

\img{90mm}{task.png}{Задача коммивояжера\label{d:task}}

В общем случае задача коммивояжера может быть сформулирована
следующим образом: найти самый выгодный маршрут, начинающийся в исходном городе и проходящий ровно один раз через каждый из указанных городов, с последующим возвратом в исходный город.


Гамильтоновым циклом называется маршрут, включающий ровно по одному разу каждую вершину графа. Таким образом, решение задачи коммивояжера -- это нахождение гамильтонова цикла минимального веса в полном взвешенном графе.

\section{Алгоритм полного перебора}

Пусть дано M — число городов, D — матрица смежности, каждый
элемент которой — вес пути из одного города в другой. Задача может быть решена перебором всех
возможных гамильтоновых циклов в заданном графе и выбором оптимального (минимального по весу). Но при таком подходе количество возможных маршрутов очень быстро возрастает с ростом M (сложность такого алгоритма составляет M! и время выполнения программы, реализующий такой подход, будет расти экспоненциально в зависимости от размеров входной матрицы). 

\section{Муравьиный алгоритм}

В основе муравьиного алгоритма лежит моделирование поведение муравьиной колонии, которое связано с распределением феромона на тропе -- ребре графа в задаче коммивояжера. При этом вероятность включения ребра в маршрут отдельного муравья пропорциональна количеству феромона на этом ребре, а количество откладываемого феромона пропорционально длине маршрута. Чем короче маршрут, тем больше феромона будет отложено на его ребрах, следовательно, большее количество муравьев будет включать его в синтез собственных маршрутов.


Моделирование такого подхода, использующего только положительную обратную связь, приводит к преждевременной сходимости — большинство муравьев двигается по локально оптимальному маршруту. Избежать, этого можно, моделируя отрицательную обратную связь в виде испарения феромона. При этом если феромон испаряется быстро, то это приводит к потере памяти колонии и забыванию хороших решений, с другой стороны, большое время испарения может привести к получению устойчивого локально оптимального решения. Теперь, с учетом особенностей задачи коммивояжера, мы можем описать локальные правила поведения муравьев при выборе пути.


У муравья есть 3 чувства: память, зрение, обоняние.


«Память» --- муравей запоминает свой маршрут. Поскольку каждый город может быть посещен только один раз, у каждого муравья есть список уже посещенных городов - список запретов. Обозначим через $J_{i,k}$ список городов, которые необходимо посетить муравью $k$, находящемуся в городе $i$;


«Зрение» --- муравей может оценить длину ребра. Видимость есть эвристическое желание посетить город $j$, если муравей находится в городе $i$. Будем считать, что видимость — величина, обратная расстоянию между городами $i$ и $j$ --- $D_{ij}$ 
	\begin{equation}
		\label{eq:vision}
		\eta_{ij} = \frac{1}{D_{ij}}
	\end{equation}


«Обоняние» --- муравей может улавливать след феромона, подтверждающий желание посетить город $j$ из города $i$, на основании опыта других муравьев. Количество феромона на ребре $(i,j)$ в момент времени $t$ обозначим через $\tau_{ij}(t)$.

На этом основании можно сформулировать вероятностно-пропорциональное правило \ref{eq:rule}, определяющее вероятность перехода $k$-ого муравья из города $i$ в город $j$:

\begin{equation}
	\label{eq:rule}
	P_{i,j}={\begin{cases}{\frac {(\tau_{i,j}^{\alpha })(\eta_{i,j}^{\beta })}{\sum (\tau_{i,j}^{\alpha })(\eta_{i,j}^{\beta })}}, & {\mbox{если ребро есть в списке городов;}}\\0,&{\mbox{иначе}}\end{cases}}
\end{equation}

\noindent где $ \tau_{i,j} - $ количество феромонов на ребре ij; $\eta_{i,j} - $расстояние от города i до j; $\alpha $ и $\beta - $ два регулируемых параметра, задающие веса следа феромона и видимости при выборе маршрута. Важным условием является то, что сумма $\alpha $ и $\beta $ должна быть постоянной.


При $\alpha=0$ алгоритм вырождается до алгоритма полного перебора (будет выбран ближайший город). Если $\beta=0$, тогда работает лишь феромонное усиление, что влечет за собой быстрое вырождение маршрутов к одному субоптимальному решению.


Пройдя ребро $(i,j)$, муравей откладывает на нем некоторое количество феромона, которое должно быть связано с оптимальностью сделанного выбора. Пусть $T_k(t)$ есть маршрут, пройденный муравьем $k$ к моменту времени t, а $L_k(t)$ --- длина этого маршрута. Пусть также $Q$ --- параметр, имеющий значение порядка длины оптимального пути. Тогда откладываемое количество феромона может быть задано формулой \ref{form:add}:

\begin{equation}\label{form:add}
	{\displaystyle \Delta \tau_{i,j}^k(t)={\begin{cases}Q/L_{k}(t), & {\mbox{если k-ый мурваей прошел по ребру ij;}}\\0,&{\mbox{иначе}}\end{cases}}}
\end{equation}

\noindent где Q - количество феромона, переносимого муравьем.\\

Правила внешней среды определяют, в первую очередь, испарение феромона. Пусть $\rho \in [0,1]$ есть коэффициент испарения, тогда правило испарения имеет вид

\begin{equation}
	\label{eq:pheromone_evaporation}
	\tau_{ij}(t+1) = (1 - \rho) * \tau_{ij}(t) + \Delta\tau_{ij}(t); \Delta\tau_{ij}(t) = \sum_{k = 1}^{m} \Delta\tau_{ij,k}(t); 
\end{equation}

\noindent где $m$ — количество муравьев в колонии.


В начале алгоритма количество феромона на ребрах принимается равным небольшому положительному числу. Общее количество муравьев остается постоянным и равным количеству городов, каждый муравей начинает маршрут из своего города. 


Сложность алгоритма: $O(t_{max} * max(m, n^2))$, где $t_{max}$ --- время жизни колонии, $m$ --- количество муравьев в колонии, $n$ --- размер графа \cite{Ulyanov}.

\section*{Вывод}

В данном разделе была рассмотрена предметная область работы и основополагающие материалы, которые в дальнейшем потребуются при реализации алгоритмов решения задачи коммивояжера.


Существуют две группы методов для решения задачи коммивояжера — точные и эвристические. К точным относится алгоритм полного перебора, к эвристическим — муравьиный алгоритм. Применение муравьиного алгоритма обосновано в тех случаях, когда необходимо быстро найти решение или когда для решения задачи достаточно получения первого приближения. В случае необходимости максимально точного рещения используется алгоритм полного перебора.
