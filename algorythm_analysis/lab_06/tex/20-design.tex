\chapter{Конструкторская часть}
В данном разделе приводится схема рассматриваемого алгоритма, определяются требования к программе.

\section{Требования к программе}

На основе изученной информации можно описать входные и выходные данные.
На вход программе должна подаваться матрица смежности либо координаты x y городов, количество городов.
На выходе программа должна выдавать найденный кратчайший путь.


Так как перевод из муравьиных дней в человеческие невозможен, выбран подход случайного леса.
Итог программы рассчитывается в муравьиных днях, несоизмеримых с человеческими. 

\section{Разработка алгоритмов}

На рисунке \ref{svgimg:ant-Page-1.svg} приведена схема муравьиного алгоритма.
На рисунке \ref{svgimg:ant-Page-2.svg} приведена схема алгоритма полного перебора.

\newpage
\svgimg{200mm}{ant-Page-1.svg}{Схема муравьиного алгоритма}
\newpage
\svgimg{200mm}{ant-Page-2.svg}{Схема алгоритма полного перебора}
\newpage

\section*{Вывод}
В данном разделе была рассмотрена схема муравьиного алгоритма, алгоритма полного перебора, были приведены требования к программе.

