\chapter{Технологическая часть}

В данном разделе будут приведены средства реализации и листинги кода.


\section{Средства реализации}

В качестве языка программирования для реализации данной лабораторной работы был выбран язык программирования Python. Выбор обусловлен наличием библиотек для измерения времени, наличием инструментов для работы с массивами.


\section{Сведения о модулях программы}
Программа состоит из следующих модулей:
main.py - главный файл программы, в котором располагается вся программа,
timer.py - файл программы c замерами времени.


\section{Реализация алгоритмов}

В листингах \ref{lst:lev_mat}, \ref{lst:dlev_rec} приведены реализации алгоритмов.
\newpage
\begin{lstlisting}[label=lst:lev_mat,caption=Муравьиный алгоритм.]
def run(self, max_iter=None):
	self.max_iter = max_iter or self.max_iter
	for i in range(self.max_iter):
		prob_matrix = (self.Tau ** self.alpha) * (self.prob_matrix_distance) ** self.beta
		for j in range(self.size_pop):
			self.Table[j, 0] = 0
			for k in range(self.n_dim - 1):
				taboo_set = set(self.Table[j, :k + 1])
				allow_list = list(set(range(self.n_dim)) - taboo_set)
				prob = prob_matrix[self.Table[j, k], allow_list]
				prob = prob / prob.sum() # нормализация вероятности
				next_point = np.random.choice(allow_list, size=1, p=prob)[0]
				self.Table[j, k + 1] = next_point
						
		y = np.array([self.func(i) for i in self.Table])
		
		index_best = y.argmin()
		x_best, y_best = self.Table[index_best, :].copy(), y[index_best].copy()
		self.generation_best_X.append(x_best)
		self.generation_best_Y.append(y_best)
		
		delta_tau = np.zeros((self.n_dim, self.n_dim))
		for j in range(self.size_pop):  # для каждого муравья
			for k in range(self.n_dim - 1):  # для каждой вершины
				n1, n2 = self.Table[j, k], self.Table[j, k + 1]
				delta_tau[n1, n2] += 1 / y[j]  # нанесение феромона
			n1, n2 = self.Table[j, self.n_dim - 1], self.Table[j, 0]
			delta_tau[n1, n2] += 1 / y[j]  # нанесение феромона
		
		self.Tau = (1 - self.rho) * self.Tau + delta_tau
		
	best_generation = np.array(self.generation_best_Y).argmin()
	self.best_x = self.generation_best_X[best_generation]
	self.best_y = self.generation_best_Y[best_generation]
	
	return self.best_x, self.best_y
\end{lstlisting}
\newpage
\begin{lstlisting}[label=lst:dlev_rec,caption= Алгоритм полного перебора.]
def full_combinations(matrix, size):
	cities = np.arange(size)
	cities_combs = []
	
	for combination in it.permutations(cities):
		cities_combs.append(list(combination))
	
	best_way = []
	min_length = float("inf")
	
	for i in range(len(cities_combs)):
		cities_combs[i].append(cities_combs[i][0])
		
		length = calc_length(matrix, size, cities_combs[i])
		
		if length < min_length:
			min_length = length
			best_way = cities_combs[i]
	
	return min_length, best_way
\end{lstlisting}
\newpage
\section{Функциональные тесты}
В таблице \ref{tabular:functional_test} приведены функциональные тесты для алгоритмов.

\begin{table}[h!]
	\captionsetup{singlelinecheck=off}\caption{\raggedright\label{tabular:functional_test} Тестирование функций}
	\begin{center}
		\begin{tabular}{c@{\hspace{7mm}}c@{\hspace{5mm}}c@{\hspace{7mm}}c@{\hspace{7mm}}}
			\hline
			Количество городов & Координаты & Ожидаемый результат \\ \hline
			\vspace{4mm}
			$
				2
			$ &
			$
				\begin{bmatrix*}
					0.0 & 0.0 \\
					0.0 & 3.0
				\end{bmatrix*}
			$ &
			$
				6
			$ \\
			\vspace{4mm}
			$
				6 
			$ &
			$
				\begin{bmatrix*}
					0.0 & 0.0 \\
					1.0 & 3.0 \\
					1.0 & 5.0\\
					0.0 & 7.0\\
					-1.0 & 5.0 \\
					-1.0 & 3.0	\\
				\end{bmatrix*}
				
			$ &
			$
				14.7
			$ \\
			\vspace{4mm}
			$
				4
			$ &
			$
			\begin{bmatrix*}
				0.0 & 0.0 \\
				0.0 & 3.0 \\
				2.0 & 0.0 \\
				2.0 & 3.0 \\
			\end{bmatrix*}
			$ &
			$
				10
			$ \\
			\vspace{4mm}
			$
				10
			$ &
			$
			\begin{bmatrix*}
				7.0 & 1.0 \\
				0.0 & 7.0\\
				0.0 & 6.0\\
				6.0 & 0.0\\
				7.0 & 0.0\\
				8.0 & 2.0\\
				6.0 & 9.0\\
				6.0 & 2.0\\
				7.0 & 9.0\\
				3.0 & 7.0\\
			\end{bmatrix*}
			$ &
			
			28.3
			 \\
			
			\vspace{4mm}
		\end{tabular}
	\end{center}
\end{table}
Алгоритмы прошли проверку.
\section*{Вывод}

Были разработан и протестирован муравьиный алгоритм, алгоритм полного перебора.