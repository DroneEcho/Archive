\chapter{Технологический раздел}
В этом разделе будет приведено описание структура программы, выбраны средства реализации ПО, приведены листинги кода, продемонстрирован интерфейс программы.
\section{Средства реализации программного обеспечения}
В качестве языка программирования для решения поставленных задач
был выбран язык программирования С++, поскольку:
\begin{itemize}
\item имеется опыт разработки на данном языке;
\item С++ обладает достаточной производительностью для быстрого исполнения трассировки лучей;
\end{itemize}

В качестве IDE была выбрана среда разработки QT Creator, так как:
\begin{itemize}
\item имеется опыт разработки с взаимодействием с данной IDE;
\item есть возможность создать графический интерфейс;
\end{itemize}
\section{Описание структуры программы}
В программе реализованы классы:
\begin{itemize}
	\item class Manager – хранит сцену, описывает методы взаимодействия со сценой и объектами на ней;
	\item class Model – описывает представление трёхмерного объекта в программе и методы работы с ним;
	\item class Light – описывает источники света и методы взаимодействия с ним;
	\item class Camera – описывает камеру и методы взаимодействия с ней;
	\item class objLoader – описывает работу с файлами расширения .obj;
	\item class Face – описывает полигоны для представления трёхмерного объекта;
	\item class Vertex – описывает вершину объекта;
	\item class Vec3, Vec4 – реализация векторов размерности 3 и 4.
	\item class Mat – описывает матрицы и методы взаимодействия с ними.
	\item class BoundingBox – описывает ограничивающий параллелепипед и методы работы с ним;
	\item class RayThread – описывает работу отдельного потока при трассировке лучей;
	\item class PixelShader – содержит функции для вычисления атрибутов объекта в конкретном пикселе
	\item class VertexShader – содержит функции для преобразования атрибутов модели при переходе к мировому пространству из объектного;
	\item class TextureShader – содержит функции для интерполяции значения текстурных координат в конкретном пикселе;
	\item class GeometryShader – содержит функции для преобразования атрибутов модели при переходе из мирового пространства в пространство нормализированных координат;
\end{itemize}
\section{Листинг кода}

На листинге \ref{lst:lev_mat0} представлен код отрисовки модели в первом режиме работы программы.
На листинге  \ref{lst:lev_mat1} представлен код закраски треугольника в первом режиме работы программы.
На листинге \ref{lst:lev_mat2} представлен код алгоритма трассировки одного луча.
\begin{lstlisting}[label=lst:lev_mat0,caption= Отрисовка модели в первом режиме работы программы]
	void Scene::rasterize(Model& _model)
	{
		auto camera = camers[curr_camera];
		auto projectMatrix = camera.projectionMatrix;
		auto viewMatrix = camera.viewMatrix();
		
		auto rotation_matrix = model.rotation_matrix;
		auto objToWorld = model.objToWorld();
		
		for (auto& face: model.faces)
		{
			auto a = vertex_shader->shade(face.a, rotation_matrix, objToWorld, camera);
			auto b = vertex_shader->shade(face.b, rotation_matrix, objToWorld, camera);
			auto c = vertex_shader->shade(face.c, rotation_matrix, objToWorld, camera);
			
			a = geom_shader->shade(a, projectMatrix, viewMatrix);
			b = geom_shader->shade(b, projectMatrix, viewMatrix);
			c = geom_shader->shade(c, projectMatrix, viewMatrix);
			rasterBarTriangle(a, b, c);
		}
	}
\end{lstlisting}

\begin{lstlisting}[label=lst:lev_mat1,caption= Закраски треугольника]
#define Min(val1, val2) std::min(val1, val2)
#define Max(val1, val2) std::max(val1, val2)
void SceneManager::rasterBarTriangle(Vertex p1_, Vertex p2_, Vertex p3_)
{
	if (!clip(p1_) && !clip(p2_) && !clip(p3_))
	{
		return;
	}
	denormolize(width, height, p1_);
	denormolize(width, height, p2_);
	denormolize(width, height, p3_);
	
	auto p1 = p1_.pos;
	auto p2 = p2_.pos;
	auto p3 = p3_.pos;
	
	float sx = std::floor(Min(Min(p1.x, p2.x), p3.x));
	float ex = std::ceil(Max(Max(p1.x, p2.x), p3.x));
	
	float sy = std::floor(Min(Min(p1.y, p2.y), p3.y));
	float ey = std::ceil(Max(Max(p1.y, p2.y), p3.y));
	
	for (int y = static_cast<int>(sy); y < static_cast<int>(ey); y++)
	{
		for (int x = static_cast<int>(sx); x < static_cast<int>(ex); x++)
		{
			Vec3f bary = toBarycentric(p1, p2, p3, Vec3f(static_cast<float>(x), static_cast<float>(y)));
			if ( (bary.x > 0.0f || fabs(bary.x) < eps) && (bary.x < 1.0f || fabs(bary.x - 1.0f) < eps) &&
			(bary.y > 0.0f || fabs(bary.y) < eps) && (bary.y < 1.0f || fabs(bary.y - 1.0f) < eps) &&
			(bary.z > 0.0f || fabs(bary.z) < eps) && (bary.z < 1.0f || fabs(bary.z - 1.0f) < eps))
			{
				auto interpolated = baryCentricInterpolation(p1, p2, p3, bary);
				interpolated.x = x;
				interpolated.y = y;
				if (testAndSet(interpolated))
				{
					auto pixel_color = pixel_shader->shade(p1_, p2_, p3_, bary) * 255.f;
					img.setPixelColor(x, y, qRgb(pixel_color.x, pixel_color.y, pixel_color.z));
				}
			}
		}
	}
	
}
\end{lstlisting}

\begin{lstlisting}[label=lst:lev_mat2,caption= Алгоритм трассировки]
Vec3f RayThread::cast_ray(const Ray &ray, int depth)
{
	
	InterSectionData data;
	if (depth > 4 || !sceneIntersect(ray, data))
	return Vec3f{0.f, 0, 0};
	
	float di = 1 - data.model.specular;
	
	float distance = 0.f;
	
	float occlusion = 1e-4f;
	
	Vec3f ambient, diffuse = {0.f, 0.f, 0.f}, spec = {0.f, 0.f, 0.f}, lightDir = {0.f, 0.f, 0.f},
	reflect_color = {0.f, 0.f, 0.f}, refract_color = {0.f, 0.f, 0.f};
	
	if (fabs(data.model.refractive) > 1e-5)
	{
		Vec3f refract_dir = refract(ray.direction, data.normal, power_ref, data.model.refractive).normalize();
		Vec3f refract_orig = Vec3f::dot(refract_dir, data.normal) < 0 ? data.point - data.normal * 1e-3f : data.point + data.normal * 1e3f;
		refract_color = cast_ray(Ray(refract_orig, refract_dir), depth + 1);
	}
	
	if (fabs(data.model.reflective) > 1e-5)
	{
		Vec3f reflect_dir = reflect(ray.direction, data.normal).normalize();
		Vec3f reflect_orig = Vec3f::dot(reflect_dir, data.normal) < 0 ? data.point - data.normal * 1e-3f : data.point + data.normal * 1e-3f;
		reflect_color = cast_ray(Ray(reflect_orig, reflect_dir), depth + 1);
	}
	
	for (auto &model: models)
	{
		if (model->isObject()) continue;
		Light* light = dynamic_cast<Light*>(model);
		if (light->t == Light::light_type::ambient)
		ambient = light->color_intensity;
		else
		{
			if (light->t == Light::light_type::point)
			{
				lightDir = (light->position - data.point);
				distance = lightDir.len();
				lightDir = lightDir.normalize();
			} else{
				lightDir = light->getDirection();
				distance = std::numeric_limits<float>::infinity();
			}
			
			auto tDot = Vec3f::dot(lightDir, data.normal);
			
			Vec3f shadow_orig = tDot < 0 ? data.point - data.normal*occlusion : data.point + data.normal*occlusion; // checking if the point lies in the shadow of the lights[i]
			InterSectionData tmpData;
			if (sceneIntersect(Ray(shadow_orig, lightDir), tmpData))
			if ((tmpData.point - shadow_orig).len() < distance)
			continue;
			
			diffuse += (light->color_intensity * std::max(0.f, Vec3f::dot(data.normal, lightDir)) * di);
			if (fabs(data.model.specular) < 1e-5) continue;
			auto r = reflect(lightDir, data.normal);
			auto r_dot = Vec3f::dot(r, ray.direction);
			auto power = powf(std::max(0.f, r_dot), data.model.n);
			spec += light->color_intensity * power * data.model.specular;
		}
		
	}
	
	return data.color.hadamard(ambient +
	diffuse +
	spec +
	reflect_color * data.model.reflective +
	refract_color * data.model.refractive).saturate();
}
\end{lstlisting}
\section{Описание интерфейса}

На рисунке \ref{img:1} представлен стартовый экран программы. Он предоставляет пользователю возможность управления камерой.


На рисунке \ref{img:2} представлена вторая вкладка программы. Она предоставляет пользователю возможность выбирать объекты и добавлять их в рабочее пространство. Каждый объект может быть модифицирован -- может быть изменен цвет или добавлена текстура. Также имеется возможность добавить источник света.


На рисунке \ref{img:3} представлена третья вкладка программы. Есть возможность изменения цвета среды.
При нажатии на кнопку ’Рендер’ программа запускает алгоритм обратной трассировки лучей. 
Все кнопки блокируются, пока алгоритм не завершит свою работу. По его окончанию на графике будет отображено реалистическое изображение.

\img{80mm}{1.png}{Интерфейс программы, страница 1 \label{img:1}}
\img{80mm}{2.png}{Интерфейс программы, страница 2 \label{img:2}}
\img{80mm}{3.png}{Интерфейс программы, страница 3 \label{img:3}}
\newpage
\section{Вывод}
Было приведено описание структура программы, выбраны средства реализации ПО, приведены листинги кода, и продемонстрирован интерфейс программы.