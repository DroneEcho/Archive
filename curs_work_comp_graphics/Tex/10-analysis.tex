\chapter{Аналитическая часть}
В этом разделе будут приведены требования к программе, описание модели трехмерного объекта в сцене, рассмотрены формализация объектов сцены. Будут также рассмотрены алгоритмы построения трехмерного изображения, модели освещения, закраски, а также способы текстуризации.
\section{Описание модели трехмерного объекта}
В качестве метода представления трехмерного объекта в сцене был выбран метод полигональной сетки.
 
\textbf{Полигональная сетка} -- совокупность вершин, рёбер и граней, которые определяют форму многогранного объекта в трехмерной компьютерной графике. 
Гранями обычно являются треугольники, четырехугольники или другие простые выпуклые многоугольники (полигоны), так как это упрощает рендеринг, но также может состоять из наиболее общих вогнутых многоугольников, или многоугольников с дырками. В данной работе было выбрано представление граней в качестве треугольников. Данный метод позволяет описывать многоугольники произвольной формы, и от количества полигонов зависит реалистичность модели, однако обработка большего количества граней влечет за собой увеличение требований к ресурсам системы, так что был выбран универсальный вариант.
\section{Формализация объектов сцены}
Сцена программы состоит из следующих объектов:
\begin{enumerate}
\item Геометрическая композиция -- представляется в виде набора геометрический объектов, описанных в одном файле. Объекты описаны в виде полигональной сетки. В качестве элементов геометрической композиции допускаются куб, параллелепипед, цилиндр, сфера, конус. Для описания каждого геометрического объекта композиции требуется указать координаты вершин, ребра и грани между ними. Далее должно быть описание группы объектов композиции с: входящими в группу объектами, цветом или текстурой объектов, свойствами поверхности (коэффициент отражения, коэффициент блеска).
\item Источник света -- представлен в виде вектора направления и обладает рядом характеристик, таких как местоположение, цвет и интенсивность излучения.
\item Камера -- описывается своими координатами расположения и направлением взгляда.
\end{enumerate}
\section{Требования к работе программы}
Программа должна обеспечивать построение реалистического изображения. Однако, стоит заметить, что программа обязана без задержек иметь возможность добавлять и удалять геометрические композиции, изменять характеристики объектов и т.д. Следовательно, программа обязана иметь два режима. Первый режим характеризуется быстродействием для удобной работы пользователя и редактирования сцены, тогда как второй режим введен для рендера реалистичных характеристик сцены.
\section{Анализ алгоритмов построения трёхмерного изображения}
	\subsection{Алгоритм Z-буфера}
	\textbf{Алгоритм Z-буфера (Z-buffer)} -- это алгоритм, использующийся в компьютерной графике для решения проблемы скрытия граней (перекрытие одной грани другой) при отображении 3D-сцен. Основная идея алгоритма заключается в том, что в каждом пикселе экрана хранится информация о глубине затравочной точки (z-координата), а также о цвете пикселя.
	
	
	Для каждого пикселя на экране алгоритм Z-буфера хранит два значения: значение z--буфера (Z--buffer), которое соответствует наиболее удаленной (глубокой) точке изображения в текущем пикселе и соответствующее значение цвета.
	
	
	При отрисовке каждого объекта в сцене для каждого пикселя на экране вычисляется его глубина (z--координата). Если z--координата текущего пикселя меньше значения z--буфера в данной точке, то значение z-координаты и цвета пикселя в z--буфере обновляются в соответствии с текущим объектом, иначе происходит игнорирование текущего пикселя. Таким образом, в конечном итоге на экране отображаются только наиболее удаленные точки изображения, то есть те, которые не скрыты другими объектами.
	
	
	Алгоритм Z-буфера является наиболее простым и эффективным методом решения проблемы скрытия по отношению ко всем другим методам. Он также обеспечивает эффективность отрисовки уровня сложности сцены и динамичности, которые могут изменяться на ходу.
	
	На рисунке \ref{z-buf} представлена иллюстрация работы алгоритма Z-буфера
	\img{70mm}{Z-buf}{\label{z-buf} Иллюстрация работы алгоритма Z-буфера}
	
	\subsection{Алгоритм обратной трассировки лучей}
	Методы трассировки лучей на сегодняшний день считаются наиболее мощными и универсальными методами создания реалистических изображений. Известно много примеров реализации алгоритмов трассировки для качественного отображения самых сложных трехмерных сцен. Можно отметить, что универсальность методов трассировки в значительной степени обусловлена тем, что в их основе лежат простые и ясные понятия, отражающие наш опыт восприятия окружающего мира.
	
	Данный метод позволяет учесть все физические явления: отражение, преломление, воссоздание теней. У алгоритма существует недостаток: трассировка лучей каждый раз начинает заново процесс вычисления цвета пиксела, рассматривая каждый луч по отдельности. Временные затраты у алгоритма существенно большие, чем у алгоритма Z-буфера.
	Алгоритм предлагает рассмотреть следующую ситуацию: через каждый пиксел изображения проходит луч, выпущенный из камеры, и программа должна определить точку пересечения этого луча со сценой. Первичный луч – луч, выпущенный из камеры.
	На рисунке \ref{tras} представлена ситуация, когда первичный луч пересекает объект в точке H1.
	\img{70mm}{trassir}{\label{tras} Иллюстрация работы алгоритма обратной трассировки лучей}
	
	Для источника света определяется, видна ли для него эта точка. Чтобы это сделать, испускается теневой луч из точки сцены к источнику. Если луч пересек какой-либо объект сцены, то значит, что точка находится в тени, и её не надо освещать. В обратном случае требуется рассчитать степень освещенности точки. Затем алгоритм рассматривает отражающие свойства объекта: если они есть, то из точки H1 выпускается отраженный луч, и процедура повторяется рекурсивно. Тоже самое происходит при рассмотрении свойств преломления. 
	\subsection{Вывод}
	Программа предоставляет два режима работы. В первом пользователь добавляет объекты и имеет возможность редактировать расположение конструкции, цвет или текстуру объектов, освещение. Во втором режиме предоставляется реалистическое изображение. Для первого режима был выбран Z-буфер так как важна скорость алгоритмов. Для второго режима был выбран алгоритм трассировки лучей так как важна реалистичность и учет оптических эффектов.
	
	\begin{tabular}{|p{5cm}|p{5cm}|p{5cm}|}
		\hline
		Характеристика & Z-буфер & Алгоритм трассировки лучей \\
		\hline
		Сложность реализации & Низкая & Высокая \\
		\hline
		Эффективность для сложных сцен & Низкая & Высокая \\
		\hline
		Точность & Высокая & Высокая \\
		\hline
		Использование ресурсов (память, процессорное время) & Высокое потребление & Низкое потребление \\
		\hline
		Поддержка прозрачности и отражений & Да & Да \\
		\hline
		Поддержка теней & Требует дополнительной обработки & Да \\
		\hline
	\end{tabular}
\section{Анализ алгоритмов закраски}
	\subsection{Метод закраски Гуро}
	Метод закраски, который основан на интерполяции интенсивности и известен как метод Гуро, позволяет устранить дискретность изменения интенсивности. Процесс закраски по методу Гуро осуществляется в четыре этапа:
	\begin{enumerate}
	\item Вычисляются нормали ко всем полигонам.
	\item Определяются нормали в вершинах путем усреднения нормалей по всем полигональным граням, которым принадлежит вершина.
	
	\img{70mm}{guro1}{Нормали к вершинам: v=(1 + 2 + 3 + 4 + v)/4}
		
	\item Используя нормали в вершинах и применяя произвольный метод закраски, вычисляются значения интенсивности в вершинах.
	\item Каждый многоугольник закрашивается путем линейной интерполяции значений интенсивностей в вершинах сначала вдоль каждого ребра, а затем и между ребрами вдоль каждой сканирующей строки.
	
	\img{70mm}{guro2}{Интерполяция интенсивностей}
	
	\end{enumerate}
	Интерполяция вдоль ребер легко объединяется с алгоритмом удаления скрытых поверхностей, построенным на принципе построчного сканирования. Для всех ребер запоминается начальная интенсивность, а также изменение интенсивности при каждом единичном шаге по координате y, Заполнение видимого интервала на сканирующей строке производится путем интерполяции между значениями интенсивности на двух ребрах, ограничивающих интервал.
	
	
	
	\begin{align}
		\begin{gathered}
		I_a = I_1\frac{y_s - y_2}{y_1 - y_2} + I_2 \frac{y_1 - y_s}{y_1 - y_2}\\
		I_b =I_1\frac{y_s - y_3}{y_1 - y_3} + I_3 \frac{y_1 - y_s}{y_1 - y_3}\\
		I_p =I_a\frac{x_b - x_y}{x_b - x_a} + I_b \frac{x_y - x_a}{x_b - x_a}
		\end{gathered}
	\end{align}
	
	Для цветных объектов отдельно интерполируется каждая из компонент цвета.
	
	Метод Гуро применим для небольших граней, расположенных на значительном расстоянии от источника света. Если же размер грани большой, то расстояние от источника света до центра будет меньше, чем до вершин, и центр грани должен выглядеть ярче, чем рёбра. Но из-за линейного закона, используемого в интерполяции, метод не позволяет это сделать, поэтому появляются участки с неестественным освещением.
	
	\subsection{Вывод}
	Вместе с алгоритмом z-буфера будет использоваться метод Гуро, так как он быстрый, и получается приемлемое качество изображения, в том числе можно заметить сглаживание тел. При создании реалистического изображения будет использоваться алгоритм обратной трассировки лучей, который не требует какого-то дополнительного алгоритма закраски.
	
	
	Гуро имеет более низкую сложность реализации и потребляет меньше ресурсов, но менее точен и не поддерживает прозрачность, отражения и объемные объекты. Алгоритм обратной трассировки лучей более эффективен для сложных сцен, более точен и поддерживает прозрачность, отражения и объемные объекты.
	
		\begin{tabular}{|p{5cm}|p{5cm}|p{5cm}|}
		\hline
		Характеристика & Гуро & Алгоритм трассировки лучей \\
		\hline
		Сложность реализации & Низкая & Высокая \\
		\hline
		Эффективность для сложных сцен & Низкая & Высокая \\
		\hline
		Точность & Низкая & Высокая \\
		\hline
		Использование ресурсов (память, процессорное время) & Низкое потребление & Низкое потребление \\
		\hline
		Поддержка прозрачности и отражений & Нет & Да \\
		\hline
		Поддержка теней & Требует дополнительной обработки & Да \\
		\hline
	\end{tabular}
\section{Анализ алгоритмов моделирования освещения}
Моделирование освещения – основополагающий элемент фотореализма.

Реалистичность изображения во многом зависит от правильного выбора алгоритма освещения. Все модели освещённости можно разделить на две группы: глобальные и локальные. Локальные модели учитывают только первичный источник света, а глобальные также рассматривают физические явления: отражение света от поверхностей, преломление света.
	\subsection{Модель Ламберта}
	В этой модели воспроизводится идеальное диффузное освещение. Свет при попадании на поверхность равномерно рассеивается во все стороны. При расчёте учитывается только ориентация нормали поверхности (нормаль $\vec{N}$ ) и направление на источник света (вектор $\vec{L}$ ). Пусть:
	
	$I_d$ – рассеянная составляющая освещённости в точке, 
	
	$K_d$ – свойство материала воспринимать рассеянное освещение, 
	
	$I_0$ – интенсивность рассеянного освещения.
	
	Тогда интенсивность можно рассчитать по формуле:
	\begin{align}
	I_d = K_d(\vec{L}, \vec{N})I_0
	\end{align}
	Модель Ламберта является одной из самых простых моделей освещения. Она часто используется в комбинации с другими моделями, так как практически в любой можно выделить диффузную составляющую. Равномерная часть освещения чаще всего представляется именно моделью Ламберта.
	\subsection{Модель Фонга}
	Модель Фонга основывается на предположении, что освещённость каждой точки можно разбить на три компоненты:
	\begin{itemize} 
	\item Фоновое освещение (ambient) -- оно присутствует на любом участке сцены, не зависит от источников света, потому считается константой;
	\item Рассеянный свет (diffffuse) -- рассчитывается также, как в модели Ламберта;
	\item Бликовая составляющая (specular) – зависит от того, насколько близко находятся вектор отражённого луча и вектор до наблюдателя. Свойства источника определяют мощность излучения для каждой из компонент, а свойства материала -- способность объекта воспринимать свет.
	\end{itemize}
	Пусть:
	
	$\vec{N}$ -- вектор нормали к поверхности в точке, 
	
	$\vec{L}$ -- падающий луч, 
	
	$\vec{R}$ -- отражённый луч, 
	
	$\vec{V}$ -- вектор, направленный к наблюдателю, 
	
	$k_a$ -- коэффициент фонового освещения, 
	
	$k_d$ -- коэффициент диффузного освещения,
	
	$k_s$ -- коэффициент зеркального освещения, 
	
	p -- степень, аппроксимирующая пространственное распределение зеркально отражённого света. 
	
	Тогда интенсивность света подсчитывается формулой:
	\begin{align}
	I_a = K_a * I_a + K_d(\underline{L}, \underline{N}) + K_s(\underline{R}, \underline{V})^p
	\end{align}
	На рисунке \ref{fon} приведён пример работы модели Фонга:
	\img{40mm}{fong}{\label{fon}Иллюстрация работы модели Фонга}
	\subsection{Модель Уиттеда}
	Эта модель освещения позволяет рассчитать интенсивность отражённого к наблюдателю луча в каждом пикселе изображения. Используемый в проекте вариант учитывает также свет от других объектов сцены или пропущенный сквозь них. 
	
	Помимо направления взгляда, отражённого луча, источника света, также рассматривается и составляющая преломления.
	Пусть:
	
	$K_a$ – коэффициент рассеянного отражения,
	
	$K_d$ – коэффициент диффузного отражения, 
	
	$K_s$ – коэффициент зеркальности, 
	
	$K_r$ – коэффициент отражения, 
	
	$K_t$ – коэффициент преломления, 
	
	$I_a$ – интенсивность фонового освещения, 
	
	$I_d$ – интенсивность для диффузного рассеивания, 
	
	$I_s$ – интенсивность для зеркальности,
	
	$I_r$ - интенсивность излучения, приходящего по отраженному лучу, 
	
	$I_t$ - интенсивность излучения, приходящего по преломленному лучу, - цвет поверхности.
	
	Тогда интенсивность по модели Уиттеда можно рассчитать по формуле:
	
	\begin{align}
		I = K_a I_a C + K_d I_d C + K_s I_s + K_r I_r + K_t I_t
	\end{align}
	\subsection{Вывод}
	Для первого режима работы лучше подходит модель Ламберта, так как она быстрее и эффективно сочетается с Z-буфером. Для создания реалистического изображения выбрана модель Уиттеда, потому что здесь в первую очередь важно качество полученного изображения, и этот способ показывает высокую эффективность в комбинации с обратной трассировкой лучей.
	
	\begin{tabular}{|p{4cm}|p{4cm}|p{4cm}|p{4cm}|}
		\hline
		Характеристика & Модель Ламберта &  Модель Уиттеда & Модель Фонга \\
		\hline
		Расчет освещенности & Используется только диффузное отражение & Используется диффузное и зеркальное отражение & Используется диффузное, зеркальное и преломленное отражение \\
		\hline
		Сложность реализации & Низкая & Средняя & Высокая \\
		\hline
		Реалистичность & Низкая & Средняя & Высокая \\
		\hline
		Использование ресурсов (память, процессорное время) & Низкое потребление &  Среднее потребление & Высокое потребление \\
		\hline
		Поддержка теней & Требует дополнительной обработки & требует дополнительной обработки & Да \\
		\hline
	\end{tabular}
\section{Текстурирование объектов трехмерной сцены}
Пусть u, v – координаты текстуры, которые требуется найти для решения задачи наложения текстур на объект трёхмерной сцены. Метод перспективно-корректного текстурирования основан на приближении u и v кусочно-линейными функциями. При отрисовке каждая сканирующая строка разбивается на части, в начале и конце каждого куска вычисляются точные значения u и v, а внутри каждой части применяется линейная интерполяция.

Пусть $S_x$ и $S_y$ – координаты, принадлежащие проекции текстурируемого треугольника. Тогда значения $\frac{1}{z}$, $\frac{u}{z}$ и $\frac{v}{z}$ линейно зависят от $S_x$ и $S_y$. Следовательно, для каждой вершины достаточно подсчитать значения  $\frac{1}{z}$, $\frac{u}{z}$ и $\frac{v}{z}$ и линейно их интерполировать.
Точные значения u и v подсчитываются по формуле:
\begin{align}
	u = \frac{(u/z)}{1/z}, v = \frac{(u/z)}{1/z}
\end{align}
\section{Алгоритм заполнения треугольника с использованием барицентрических координат}
\textbf{Барицентрические координаты} -- это координаты, в которых точка треугольника описывается как линейная комбинация вершин.

В работе используется нормализованный вариант: суммарный вес трёх вершин равен единице:
\begin{align}
	\begin{gathered}
	p = b_o v_0 + b_1 v_1 + b_2 v_2\\
	b_0 + b_1 + b_2 = 1
	\end{gathered}
\end{align}
Такой выбор обусловлен тем, что барицентрические координаты легко вычислить, так как они равны отношению площадей треугольников, которые образует точка внутри треугольника и вершина, к общей площади треугольника.

Третья координата вычисляется через свойство нормировки, то есть фактически имеется только две степени свободы.

Барицентрические координаты позволяют интерполировать значение любого атрибута в произвольной точке треугольника: значение атрибута в заданной точке треугольника равно линейной комбинации барицентрических координат и значений атрибута в соответствующих вершинах:
\begin{align}
T = T_0 b_0 + T_1 b_1 + T_2 b_2
\end{align}
Алгоритм закраски с использованием барицентрических координат состоит из двух этапов:
\newpage
\begin{enumerate}
\item Поиск прямоугольника, минимального по площади, чтобы он содержал в себе рассматриваемый треугольник.
\item Сканирование прямоугольника слева направо и вычисление барицентрических координат для каждого выбранного пикселя. Если значение координат находится в промежутке от 0 до 1, и сумма по всем координатам не превышает 1, то пиксель закрашивается.
\end{enumerate}
\section{Вывод}
Было приведено описание модели трёхмерного объекта в сцене, рассмотрены формализация объектов сцены и требования к работе программы. Рассмотрены алгоритмы построения трёхмерного изображения, методы закраски, модели освещения, а также способы текстуризации и закраски треугольников.