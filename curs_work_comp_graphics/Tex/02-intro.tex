\chapter*{Введение}
\addcontentsline{toc}{chapter}{Введение}

В настоящее время вопросы, связанные с отображением на экране дисплея разнообразных изображений, как никогда актуальны. Графика используется практически во всех областях деятельности человека, так или иначе связанных с использованием компьютера. Графически представленная информация является как удобным средством для взаимодействия с пользователем, так и неотъемлемой частью вычислительного комплекса, таких как: моделирование сложных процессов, природных явлений, реалистичной графики в трехмерных компьютерных играх.

До недавнего времени основным критерием выбора способа отображения трехмерных объектов являлась скорость вычислений, в силу того, что мощности компьютеров не хватало для полноценной реализации существующих алгоритмов. Однако сейчас технологии продвинулись далеко вперед и появилась потребность в отображении реалистичных изображений с отображением таких оптических эффектов, как отражение, преломление, блики света от воды и т.д.

Актуальность работы обусловлена необходимостью создания реалистических изображений аппаратными методами, используя оптимальные для этой задачи алгоритмы.


Цель курсовой работы: разработать программу для проектирования изображения модели зàмка из композиции трехмерных геометрических объектов: куб, параллелепипед, цилиндр, сфера, конус. Предусмотреть возможность изменения пользователем через интерфейс программы:
\begin{itemize}
	\item Цвета, текстуры объектов сцены;
	\item Координат и направления камеры;
	\item Интенсивности и цвета источника освещения.
\end{itemize}
\newpage
Для достижения поставленной цели, требуется выполнить следующие задачи:
\begin{enumerate}
	\item Провести анализ существующих алгоритмов для реализации и решения задачи создания конструктора из примитивных фигур и их реалистическое представление, выбрать оптимальные;
	\item Реализовать выбранные алгоритмы, создать программный продукт для решения поставленной задачи проектирования изображения модели зàмка;
	\item Реализовать интуитивный интерфейс для удобства пользователя;
	\item Провести исследования скорости выполнения выбранных алгоритмов.
\end{enumerate}
